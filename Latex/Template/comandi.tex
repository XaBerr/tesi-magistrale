%*****************
%* nuovi comandi *
%*****************

\newcommand{\abs}[1]{\left|#1\right|}                               % modulo
\newcommand{\dato}{\left|\right.}                                   % probabilit\`{a} condizionata
\newcommand{\fun}[1]{\mathrm{#1}}                                   % stile funzione
\newcommand{\imp}{\;\;\Longrightarrow\;\;}                          % implicazione
\newcommand{\norma}[1]{\left\| #1 \right\|}                         % norma
\newcommand{\prob}[1]{\mathrm{P}\!\left[#1\right]}                  % probabilit\`{a}
\newcommand{\expect}[1]{\mathrm{E}\!\left[#1\right]}                % aspettazione
\newcommand{\sse}{\;\;\Longleftrightarrow\;\;}                      % se e solo se
\newcommand{\vect}[1]{{\boldsymbol{\mathrm{#1}}}}                   % stile vettore
\newcommand{\real}[1]{\fun{Re}\left[#1\right]}                      % parte reale
\newcommand{\imag}[1]{\fun{Im}\left[#1\right]}                      % parte immaginaria
\newcommand{\Dim}[1]{\fun{dim}\left[#1\right]}                      % dimensione di una matrice
\newcommand{\Det}[1]{\fun{det}\left[#1\right]}                      % determinante di una matrice
\newcommand{\Ker}[1]{\fun{ker}\left[#1\right]}                      % ker di una matrice
\newcommand{\rango}[1]{\fun{rango}\left[#1\right]}                  % rango di una matrice
\newcommand{\scalare}[2]{\left\langle #1, #2 \right\rangle}         % prodotto scalare
\newcommand{\blbrace}{\left  \lbrace}                               % parentesi graffa sinistra grande
\newcommand{\brbrace}{\right \rbrace}                               % parentesi graffa destra grande
\newcommand{\sinc}{\fun{sinc}}                                      % sinc
\newcommand{\rect}{\fun{rect}}                                      % rect
\newcommand{\rcos}{\fun{rcos}}                                      % rcos
\newcommand{\sgn}{\fun{sgn}}                                        % sgn
\newcommand{\N}{\mathbb{N}}                                         % insieme dei numeri naturali
\newcommand{\Z}{\mathbb{Z}}                                         % insieme dei numeri interi
\newcommand{\Q}{\mathbb{Q}}                                         % insieme dei numeri razionali
\newcommand{\R}{\mathbb{R}}                                         % insieme dei numeri reali
\newcommand{\C}{\mathbb{C}}                                         % insieme dei numeri complessi
\newcommand{\seq}[2][n]{#2_{0}, #2_{1}, \ldots, \, #2_{#1}}         % sequenza
\newcommand{\Span}[2][n]
{\fun{span} \blbrace #2_{1}, #2_{2}, \ldots, \, #2_{#1} \brbrace}   % spazio generato
\newcommand{\ddt}{\frac{\fun{d}}{\fun{dt}}}                         % derivata
\newcommand{\Div}[2]{#1 \; \mid \; #2}                              % divide
\newcommand{\MCD}[2]{\fun{MCD}\(#1, #2\)}                           % massimo comun divisore
\newcommand{\mcm}[2]{\fun{mcm}\(#1, #2\)}                           % minimo comune multiplo
\newcommand{\goodgap}{
  \hspace{\subfigtopskip}
  \hspace{\subfigbottomskip}
}                                                                   % interlinea opportuna per le sottofigure
\newcommand{\eng}[1]{\emph{#1}}                                     % inglese
\newcommand{\virg}[1]{``#1"}                                        % fa una citazione tra virgolette
\newcommand{\textttvar}[1]{{\ttvar #1}}

% Impostazioni del codice
\definecolor{darkgreen}{rgb}{0, 0.6, 0}
\definecolor{gray}{rgb}{0.5, 0.5, 0.5}
\definecolor{lightgray}{rgb}{0.8, 0.8, 0.8}
\definecolor{mauve}{rgb}{0.58, 0, 0.82}
\definecolor{blue}{rgb}{0, 0, 0.82}
\definecolor{black}{rgb}{0, 0, 0}

\lstset{
  frame=tb,
  language=c++,
  aboveskip=3mm,
  belowskip=3mm,
  showstringspaces=false,
  columns=flexible,
  basicstyle={\small\ttfamily},
  numbers=none,
  numberstyle=\tiny\color{gray},
  keywordstyle=\color{blue},
  commentstyle=\color{lightgray},
  stringstyle=\color{darkgreen},
  breaklines=true,
  breakatwhitespace=true,
  tabsize=3
}

\lstdefinelanguage{javascript}{
  keywords={typeof, new, true, false, catch, function, return, null, catch, switch, var, if, in, while, do, else, case, break},
  keywordstyle=\color{blue}\bfseries,
  ndkeywords={class, export, boolean, throw, implements, import, this},
  ndkeywordstyle=\color{gray}\bfseries,
  identifierstyle=\color{black},
  sensitive=false,
  comment=[l]{//},
  morecomment=[s]{/*}{*/},
  commentstyle=\color{lightgray}\ttfamily,
  stringstyle=\color{darkgreen}\ttfamily,
  morestring=[b]',
  morestring=[b]"
}


\lstdefinelanguage{cmake}{
  keywords={SET, find_path, find_library, include, find_package_handle_standard_args, HINTS, NAMES, TRUE, FALSE, ON, OFF, message, mark_as_advanced, DEFAULT_MSG, PRIVATE, PUBLIC, \$, find_package, target_include_directories, target_link_libraries},
  keywordstyle=\color{blue}\bfseries,
  ndkeywords={class, export, boolean, throw, implements, import, this},
  ndkeywordstyle=\color{gray}\bfseries,
  identifierstyle=\color{black},
  sensitive=false,
  comment=[l]{\#},
  morecomment=[s]{/*}{*/},
  commentstyle=\color{lightgray}\ttfamily,
  stringstyle=\color{darkgreen}\ttfamily,
  morestring=[b]',
  morestring=[b]"
}

%****************************
%* ridefinizioni di comandi *
%****************************

\renewcommand{\(}{\left(}                                     % parentesi tonda sinistra grande
\renewcommand{\)}{\right)}                                    % parentesi tonda destra grande
\renewcommand{\[}{\left[}                                     % parentesi quadra sinistra grande
\renewcommand{\]}{\right]}                                    % parentesi quadra destra grande
\renewcommand{\exp}[1]{\fun{e}^{#1}}                          % esponenziale
\renewcommand{\gcd}[2]{\fun{gcd}\(#1, #2\)}                   % massimo comun divisore

\renewcommand{\lstlistingname}{Codice}
\renewcommand{\lstlistlistingname}{Elenco dei listati codice}


% ALTRI
\newcommand{\idx}[1]{\index{#1}\textbf{#1}}
\newcommand{\lvli}[1]{\clearpage{\pagestyle{plain}\cleardoublepage}\chapter{#1}}
\newcommand{\lvlii}[1]{\section{#1}\label{sec:#1}}
\newcommand{\lvliii}[1]{\subsection{#1}\label{sec:#1}}
\newcommand{\lvliv}[1]{\subsubsection{#1}\label{sec:#1}}
\newcommand{\lvlv}[1]{\paragraph{#1}\label{sec:#1}\mbox{}\\}
\newcommand{\lvlvi}[1]{\subparagraph{#1}\label{sec:#1}}

\newcommand{\img}[2] {
  \begin{figure}[H]
    \includegraphics[width=\linewidth]{img/#1}
    \caption{#2}
    \label{fig:#1}
  \end{figure}
}

\newcommand{\imgs}[3] {
  \begin{figure}[H]
    \centering
    \includegraphics[scale=#1]{img/#2}
    \caption{#3}
    \label{fig:#2}
  \end{figure}
}

\newcommand{\imgTwo}[6]{
  \begin{figure}[H]
    \minipage{0.5\textwidth}
    \centering
    \includegraphics[scale=#1]{img/#2}
    \caption{#3}
    \label{fig:#2}
    \endminipage\hfill
    \minipage{0.5\textwidth}
    \centering
    \includegraphics[scale=#4]{img/#5}
    \caption{#6}
    \label{fig:#5}
    \endminipage
  \end{figure}
}

\newcommand{\imgThree}[9]{
  \begin{figure}[H]
    \minipage{0.32\textwidth}
    \centering
    \includegraphics[scale=#1]{img/#2}
    \caption{#3}
    \label{fig:#2}
    \endminipage\hfill
    \minipage{0.32\textwidth}
    \centering
    \includegraphics[scale=#4]{img/#5}
    \caption{#6}
    \label{fig:#5}
    \endminipage\hfill
    \minipage{0.32\textwidth}%
    \centering
    \includegraphics[scale=#7]{img/#8}
    \caption{#9}
    \label{fig:#8}
    \endminipage
  \end{figure}
}

\newcommand{\fig}[1] {Fig. \ref{fig:#1}}
\newcommand{\sect}[1] {Sec. \ref{sec:#1}}