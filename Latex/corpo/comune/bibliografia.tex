\newpage
\begin{thebibliography}{99}

  \bibitem{a1}
  Terhal,
  B. M. Quantum supremacy,
  here we come. Nature Physics 14,
  530–531 (2018).

  \bibitem{a2}
  Lutchyn,
  R. M. et al. Majorana zero modes in superconductor-semiconductor heterostructures. Nature Reviews Materials 3,
  52–68 (2018).

  \bibitem{a3}
  Johnson,
  M. W. et al. Quantum annealing with manufactured spins. Nature 473,
  194–198 (2011).

  \bibitem{a4}
  Rønnow,
  T. F. et al. Defining and detecting quantum speedup. Science 345,
  420–424 (2014).

  \bibitem{a24}
  Valerio Scarani et al.
  The security of practical quantum key distribution,
  Rev. Mod. Phys. 81, 1301,
  (2009).

  \bibitem{a25}
  Nicolas Gisin et al.
  Quantum cryptography,
  (2002).

  \bibitem{a5}
  Ekert,
  A. K. Quantum Cryptography Based on Bell’s Theorem. 413–418 (1992) doi:10.1007/978-1-4615-3386-3\_34.

  \bibitem{a6}
  Bennett,
  C. H. \& Brassard,
  G. Quantum cryptography: Public key distribution and coin tossing. Theoretical Computer Science 560,
  7–11 (2014).

  \bibitem{a7}
  Lo,
  H. K.,
  Curty,
  M. \& Qi,
  B. Measurement device independent quantum key distribution. Physical Review Letters 108,
  1–7 (2012).

  \bibitem{a8}
  Pawłowski,
  M. \& Brunner,
  N. Semi-device-independent security of one-way quantum key distribution. Physical Review A - Atomic,
  Molecular,
  and Optical Physics 84,
  1–5 (2011).

  \bibitem{a9}
  Bennett,
  C. H. Quantum cryptography using any two nonorthogonal states. Physical Review Letters 68,
  3121–3124 (1992).

  \bibitem{a10}
  Kato,
  G. \& Tamaki,
  K. Security of six-state quantum key distribution protocol with threshold detectors. Scientific Reports 6,
  1–5 (2016).

  \bibitem{a11}
  Acín,
  A.,
  Gisin,
  N. \& Masanes,
  L. From Bell’s theorem to secure quantum key distribution. Physical Review Letters 97,
  1–4 (2006).

  \bibitem{a12}
  Calderaro,
  L. et al. Towards quantum communication from global navigation satellite system. Quantum Science and Technology 4,
  (2019).

  \bibitem{a13}
  Vallone,
  G. et al. Interference at the Single Photon Level Along Satellite-Ground Channels. Physical Review Letters 116,
  1–6 (2016).

  \bibitem{a14}
  Vallone,
  G. et al. Experimental Satellite Quantum Communications. Physical Review Letters 115,
  1–5 (2015).

  \bibitem{a15}
  Ji,
  W. Strategic Priority Program on Space Science. 34,
  505–515 (2015).

  \bibitem{a16}
  Bedington,
  R. et al. Nanosatellite experiments to enable future space-based QKD missions. EPJ Quantum Technology 3,
  (2016).

  \bibitem{a17}
  Naughton,
  D. et al. Design considerations for an optical link supporting intersatellite quantum key distribution. Optical Engineering 58,
  1 (2019).

  \bibitem{a18}
  Neumann,
  S. P. et al. Q 3 Sat: Quantum communications uplink to a 3U CubeSat—feasibility \& design. EPJ Quantum Technology 5,
  1–24 (2018).

  \bibitem{a19}
  Kimble,
  H. J. The quantum internet. Nature 453,
  1023–1030 (2008).

  \bibitem{a20}
  Benenti Giuliano,
  Principles Of Quantum Computation,
  (2004).

  \bibitem{a21}
  David M. Pozar,
  Microwave Engineering,
  (1990).

  \bibitem{a22}
  Peter C. et al.
  Linear Optical Quantum Computing in a Single Spatial Mode,
  (2014).

  \bibitem{a23}
  P. Grangier et al.
  Experimental evidence for a photon anticorrelation effect on a beam splitter: A new light on single-photon interferences,
  (1987).

  \bibitem{b24}
  Xiongfeng Ma et al.
  Practical decoy state for quantum key distribution,
  Rev. Mod. Phys. 72, 012326,
  (2005).

\end{thebibliography}
