\lvliv{Configuring your root CMakeLists.txt}

This project builds two libraries called BoilerplateQt and
BoilerplateQtGui. To import these libraries in your personal project you
can do the following. Assuming your project structure is based on
BoilerplateQt, you can add to the \texttt{config.json} the following
lines:

\begin{lstlisting}[language=javascript, gobble=2]
  "modules": [
    {
      "other module": "info"
    },
    {
      "name": "BoilerplateQt",
      "path": "gitlab.dei.unipd.it/Calderaro/BoilerplateQt.git",
      "tag": "1.0.0",
      "avaiable": "YES",
      "getGui": "YES"
    }
  ]
\end{lstlisting}

\textbf{Keep in mind that this will force your submodules to use this version (the one described in the ``tag'') of BoilerplateQt. Make sure that your submodule works fine with the version you set.}

This will add two library targets: BoilerplateQt and BoilerplateQtGui.
Other options:

\begin{itemize}
  \item
        If you do not want to import the BoilerplateQtGui library set
        \texttt{"getGui":\ "NO"}
\end{itemize}

\begin{itemize}
  \item
        You can just force your submodules to use a particular version of the
        BoilerplateQt libraries by setting \texttt{"available":\ "NO"}. Your
        project will not import the BoilerplateQt and BoilerplateQtGui
        targets.
\end{itemize}

\lvliv{Add the BoilerplateQtWidget to your GUI
  Widget}

Show the BoilerplateQtWidget inside your GUI Widget following these
steps:

\begin{itemize}
  \item
        create an empty Widget in your GUI Widget and name it as
        ``boilerplateQtWidget''.
\end{itemize}

\begin{itemize}
  \item
        right-click on the boilerplateQtWidget and select ``promote to''.
\end{itemize}

\begin{itemize}
  \item
        Use ``QWidget'' as base class.
\end{itemize}

\begin{itemize}
  \item
        In ``Promoted class name'' insert ``BoilerplateQtWidget''.
\end{itemize}

\begin{itemize}
  \item
        In ``Header file'' insert
        \textless BoilerplateQt/gui/boilerplateqtwidget.h\textgreater.
\end{itemize}

Connect BoilerplateQt business logic to the presentation of
BoilerplateQt Widget following these steps:

\begin{itemize}
  \item
        add two private pointers of type BoilerplateQt and SimpleLabel to your
        GUI Widget class.
\end{itemize}

\begin{lstlisting}[language=c++, gobble=2]
  BoilerplateQt *boilerplateQt;
  SimpleLabel *simpleLabel;
\end{lstlisting}

\begin{itemize}
  \item
        assign two objects to the pointers in a method of your GUI Widget
        class.
\end{itemize}

\begin{itemize}
  \item
        call the setBoilerplateQt method of BoilerplateQtWidget, passing the
        two pointers as argument:
\end{itemize}

\begin{lstlisting}[language=c++, gobble=2]
  ui->boilerplateQtWidget->setBoilerplateQt(boilerplateQt, simpleLabel).
\end{lstlisting}
