\lvli{PAT software design}
\lvlii{The components analysis}
The aim of this project was to develop modular software that could control a PAT system. The PAT system had to be able to align the telescope with different light sources such as the light of a star or the laser of another telescope.
In addition, the PAT system had to be able to control all devices in the \sect{Components}.
Starting from the PAT model \fig{PAT} a simplified model has been developed \fig{simplified-model} to highlight the various components to be implemented.
In questo modello si vede che tre sono le principali famiglie da implementare: sensori, controllori, e attuatori. Le frecce in questo schema indicano i passaggi di informazione, in altre parole, sensore a attuatore dovranno comunicare con il controllore e vice versa.
It is therefore necessary to establish a common interface between the various sensors and actuators to connect them with the controllers. The study of these interfaces required multiple iterations to ensure maximum flexibility.
\img{simplified-model}{Simplified model for PAT system.}

\lvlii{The GUI analysis}
To get a complete idea of the goal of the code the next step was to study what are the possible setups and what are the features that a user expects. For this reason, an analysis of the graphical interfaces and the procedures to be carried out for alignment was done.
In this section will be reported the global interface that will be an assembly of the graphics of the individual components.

\lvliii{Possible setups}
In this system there are three basic setups:
\textbf{mirror-camera-pc}, \textbf{skywatcher-camera-pc}, and \textbf{mirror-PSD-KPA101}.

\lvliv{Mirror-camera-pc}
In this setup the pc is connected with two devices: the first is
the KPA101 used to control the mirror; the second is the camera used as
sensor. Here two types of controllers are available: \textbf{manual},
\textbf{calibrated}. Manual give the possibility to move the mirror using a joystick taking
in input times and voltages. Calibrated has two functionalities: the first is to use a joystick
similar to manual that it work taking in input a distance in pixels; the
second is to use and automatic controller that align the centroid to the
target that it is called \textbf{one shot}.

\lvliv{Skywatcher-camera-pc}
In this setup the pc is connected with two devices: the first is
the skywatcher mout used to control the mirror; the second is the camera used as
sensor. The controllers available are the same of the mirror-camera-pc setup.

\lvliv{Mirror-PSD-KPA101}
In this setup the pc is connected only to the KPA101, the KPA101 is
then connected to the mirror and to the PSD. Here three types of controllers are available: \textbf{manual}, \textbf{calibrated}, \textbf{PIDKPA101}. Manual and calibrated are the same as the previous point. PIDKPA101 use the integrated PID-controller available in the KPA101.

\lvliii{Possible GUIs}

\img{gui}{Main GUI of the program.}

The GUI can be divided in two parts: the left side show the sensor
information, the right side show the controllers information.

In the sensor's side is composed by (form the top):

\begin{itemize}
  \itemsep1pt\parskip0pt\parsep0pt
  \item
        the frame rendering that contains three objects:
  \item
        the red dots represents the centroid of the beam;
  \item
        the red square represents the area where the controller calculates the
        centroid;
  \item
        the green circle represents the target;
  \item
        the set of parameters of the camera/PSD;
  \item
        the set of parameters to position the target;
  \item
        the plotter that show in real time the distances in pixels between the
        target and the centroid.
\end{itemize}

The controller's side there are three tabs: manual controller, calibrated
controller and PIDKPA101 controller (only for PSD). The Manual
controller and the PID KPA controller are simpler than calibrated. In the manual
controller voltages and times are the two parameters to be configured to optimize the joystick that will be use for the movement.

In the PIDKPA101 after the configuration of the PID parameters a checkbox \emph{enable PID}
is avaliable to start the controller. The \emph{auto open closed loop} checkbox is used to
disable the controller when the intensity of the beam go outside the
threshold range.

The calibrated controller is more complex so a separate
section is reserved for it.

\lvliv{Calibrated controller}

\imgThree{0.3}{calibrated1}{Controller interface.}{0.3}{calibrated2}{Analyzer interface.}{0.3}{calibrated3}{Graph interface.}

The aim of this controller is to be able to use the \emph{One shot
  controller}, but before to be able to use it a calibration is required. The calibration procedure can be decomposed into two parts:
data acquisition, data analysis.

\paragraph{Data acquisition}

In this phase several calibration points will be taken and stored into a file. The procedure
is the following:

\begin{itemize}
  \itemsep1pt\parskip0pt\parsep0pt
  \item
        in the \emph{Analyzer} tab set the voltages and the time interval;
  \item
        autogenerate the filename;
  \item
        start the point acquisition.
\end{itemize}

\paragraph{Data analysis}

In this phase the calibration points storde into a
file will be analyzed to retrieve the low/high voltages configuration. The procedure is the following:

\begin{itemize}
  \itemsep1pt\parskip0pt\parsep0pt
  \item
        in the \emph{Controller} tab select the file and press \emph{Upload
          and Calibrated};
  \item
        go into \emph{Graphs} tab and check that the analysis is completed
        correctly;
  \item
        if everything is ok then go into the \emph{Controller} tab and set the
        configuration to the low voltages or high voltages pressing the
        corresponding button.
\end{itemize}

\paragraph{One shot controller}

After the selection of both low and high voltages, the one shot checkbox is ready to be pressed in the \emph{Controller} tab. To
swap between high and low voltages a time threshold is available in the same tab. If the movement is too long a high voltages will be selected to reduce the time, if not a low voltages will be selected to increase the precision.

\lvlii{The repositories structure}
\img{repositories}{This is a screenshot of all the repositories developed in the PAT group.}
Starting from the simplified model \fig{simplified-model} has been chosen to create: a single repository for all controllers (Controller), a single repository for all actuators (Mirror), and two separate repositories for sensors (Camera, PSD).
As a communication interface between Controller and Mirror was made a Mirror class. As a communication interface between the Controller class and Camera/PSD it was chosen to create a Centroid class stored in a separate repository (Centroid).
The KPA101 repository was added to contain the KPA101 driver, while the Utils repository was created to contain a list of classes useful to all other repositories.
The Setup repository was created with the idea of taking pieces from the various repositories and composing setups, in other words, it contains a list of possible PAT combinations.
The last two repositories that have been created are: Docs that contains quick documentation for programmers who want to contribute to the PAT and PSD-bias that contains the software created to control a circuit that will be explained in the next chapter.

\lvlii{The repositories documentation}
This section lists all the documentation of the developed repositories.

  {\parindent0pt
    \lvliii{Utils}

This project contains a set of utilities for the PAT's repositories.
This REPO contains:

\begin{itemize}
  \tightlist
  \item
        Interval
  \item
        Point
  \item
        Range
  \item
        BoundedParameter
\end{itemize}

\textbf{Requirements}

\includegraphics[scale=0.7]{img/shilds/cpp.png}
\includegraphics[scale=0.7]{img/shilds/cmake.png}
\includegraphics[scale=0.7]{img/shilds/git.png}
\includegraphics[scale=0.7]{img/shilds/doxygen.png}
\includegraphics[scale=0.7]{img/shilds/sphinx.png}
\includegraphics[scale=0.7]{img/shilds/win.png}
\includegraphics[scale=0.7]{img/shilds/mac.png}
\includegraphics[scale=0.7]{img/shilds/linux.png}

\lvliv{Generality}

\lvlv{Import}

Import as an external library into your project by copy-paste the
following lines in your \texttt{config.json}.

\begin{lstlisting}[language=javascript, gobble=2]
  {
    "name"     : "PATUtils",
    "path"     : "gitlab.dei.unipd.it/PAT/Utils.git",
    "tag"      : "HEAD",
    "available": "YES",
    "getGui"   : "NO"
  }
\end{lstlisting}


\lvlv{Prerequisites}

These libraries need to be installed manually in your system:

\begin{itemize}
  \tightlist
  \item
        \href{https://www.qt.io/}{Qt} 5.14.2
\end{itemize}

The library documentation is generated through
\href{http://www.doxygen.nl/download.html}{Doxygen 1.8.13}. Additional
documentation in the \texttt{index} folder is generated through the
\href{https://www.anaconda.com/products/individual}{python3} package
\href{https://www.sphinx-doc.org/en/master/}{Sphinx} using the following
extensions (which you can install through pip3):

\begin{itemize}
  \tightlist
  \item
        \href{https://pypi.org/project/Sphinx/}{Sphinx 3.0.2}
  \item
        \href{https://sphinx-rtd-theme.readthedocs.io/en/stable/}{Sphinx read
          the doc theme} (to use the read the doc theme for html documentation)
  \item
        \href{https://pypi.org/project/breathe/}{Breathe} (to use the xml
        output of doxygen)
  \item
        \href{https://pypi.org/project/sphinx-markdown-builder/}{Sphinx-markdown-builder}
        (to generate the markdown version for gitlab wiki)
\end{itemize}

\texttt{pip3\ install\ Sphinx\ sphinx\_rtd\_theme\ breathe\ sphinx-markdown-builder}

\lvliv{Usage}

\lvlv{Interval}

Interval give the possibility to generate a base-type structure that it
has a \textbf{min} and \textbf{max} value.

First you need to include the header file and use the namespace.

\begin{lstlisting}[language=c++, gobble=2]
  #include <PAT/Utils/Interval.h>
  using namespace PAT::Utils;
\end{lstlisting}

Then you can define the variable ready to be used.

\begin{lstlisting}[language=c++, gobble=2]
  Interval<double> interval{0, 100};
  interval.min = 0;
  interval.max = 100;
\end{lstlisting}

\lvlv{Point}

Point give the possibility to generate a 2d-point that it has a
\textbf{x} and \textbf{y} value.

First you need to include the header file and use the namespace.

\begin{lstlisting}[language=c++, gobble=2]
  #include <PAT/Utils/Point.h>
  using namespace PAT::Utils;
\end{lstlisting}

Then you can define the variable ready to be used.

\begin{lstlisting}[language=c++, gobble=2]
  Point<double> point{0, 100};
  point.x = 0;
  point.y = 100;  
\end{lstlisting}

It has some operations already defined suc as: sum, division, scalar
product.

\begin{lstlisting}[language=c++, gobble=2]
  Point<double> point2{0, 100};
  point += point2;
  point -= point2;
  point /= 3;
  point *= 3;  
\end{lstlisting}

\lvlv{Rectangle}

Rectangle give the possibility to generate a rectangular range defined
by two points(\textbf{Point}) that they are \textbf{start} and
\textbf{end}.

First you need to include the header file and use the namespace.

\begin{lstlisting}[language=c++, gobble=2]
  #include <PAT/Utils/Rectangle.h>
  using namespace PAT::Utils;
\end{lstlisting}

Then you can define the variable ready to be used.

\begin{lstlisting}[language=c++, gobble=2]
  Rectangle<double> rectangle{{0, 0}, {100, 100}};
  rectangle.start.x = 0;
  rectangle.start.y = 0;
  rectangle.end.x = 100;
  rectangle.end.y = 100;
  Point<double> point{50, 50};
  if(rectangle.isInside(point))
   rectangle.end = point;
\end{lstlisting}

\lvlv{Circle}

Circle give the possibility to generate a circular range defined by a
central point(Point) and a radius.

First you need to include the header file and use the namespace.

\begin{lstlisting}[language=c++, gobble=2]
  #include <PAT/Utils/Circle.h>
  using namespace PAT::Utils;
\end{lstlisting}

Then you can define the variable ready to be used.

\begin{lstlisting}[language=c++, gobble=2]
  Circle<double> circle{{0, 0}, 100};
  circle.center.x = 0;
  circle.center.y = 0;
  circle.radius = 100;
  Point<double> point{50, 50};
  if(circle.isInside(point))
    circle.radius = point.x;
  Rectangle<double> boundaries = circle.getBoundaries();
\end{lstlisting}

\lvlv{BoundedParameter}

BoundedParameter give the possibility to define a parameter that it has:
\textbf{min} value, \textbf{max} value, \textbf{increment} value, and
\textbf{value} value.

First you need to include the header file and use the namespace.

\begin{lstlisting}[language=c++, gobble=2]
  #include <PAT/Utils/BoundedParameter.h>
  using namespace PAT::Utils;
\end{lstlisting}

Then you can define the variable ready to be used.

\begin{lstlisting}[language=c++, gobble=2]
  BoundedParameter<double> parameter{0, 100, 1, 50};
  parameter.min = 0;
  parameter.max = 100;
  parameter.increment = 1;
  parameter.value = 50;
\end{lstlisting}

\textbf{Warning}

The order of these parameters is imporant because is the same of the IDS
camera (min, max, increment).

\lvliv{UML}

\img{utils}{Utils UML.}

    \lvliii{Centroid}

This project contains the centroid class. This class contains the
centroid information shared between the sensor and the controller in the
PAT-system.

\textbf{Requirements}

\includegraphics[scale=0.7]{img/shilds/cpp.png}
\includegraphics[scale=0.7]{img/shilds/cmake.png}
\includegraphics[scale=0.7]{img/shilds/git.png}
\includegraphics[scale=0.7]{img/shilds/doxygen.png}
\includegraphics[scale=0.7]{img/shilds/sphinx.png}
\includegraphics[scale=0.7]{img/shilds/win.png}
\includegraphics[scale=0.7]{img/shilds/mac.png}
\includegraphics[scale=0.7]{img/shilds/linux.png}

\lvliv{Generality}

\lvlv{Import}

Import as an external library into your project by copy-paste the
following lines in your \texttt{config.json}.

\begin{lstlisting}[language=javascript, gobble=2]
  {
    "name"     : "PATCentroid",
    "path"     : "gitlab.dei.unipd.it/PAT/Centroid.git",
    "tag"      : "HEAD",
    "available": "YES",
    "getGui"   : "NO"
  }
\end{lstlisting}

\lvlv{Prerequisites}

The following libraries are auto fetched from the gitlab.dei.unipd.it
host (ask the owner of this repo to become a member):

\begin{itemize}
  \tightlist
  \item
        \href{https://gitlab.dei.unipd.it/PAT/Utils.git}{Utils} 1.0.0
\end{itemize}

These other libraries need to be installed manually in your system:

\begin{itemize}
  \tightlist
  \item
        \href{https://www.qt.io/}{Qt} 5.14.2
\end{itemize}

The library documentation is generated through
\href{http://www.doxygen.nl/download.html}{Doxygen 1.8.13}. Additional
documentation in the \texttt{index} folder is generated through the
\href{https://www.anaconda.com/products/individual}{python3} package
\href{https://www.sphinx-doc.org/en/master/}{Sphinx} using the following
extensions (which you can install through pip3):

\begin{itemize}
  \tightlist
  \item
        \href{https://pypi.org/project/Sphinx/}{Sphinx 3.0.2}
  \item
        \href{https://sphinx-rtd-theme.readthedocs.io/en/stable/}{Sphinx read
          the doc theme} (to use the read the doc theme for html documentation)
  \item
        \href{https://pypi.org/project/breathe/}{Breathe} (to use the xml
        output of doxygen)
  \item
        \href{https://pypi.org/project/sphinx-markdown-builder/}{Sphinx-markdown-builder}
        (to generate the markdown version for gitlab wiki)
\end{itemize}

\texttt{pip3\ install\ Sphinx\ sphinx\_rtd\_theme\ breathe\ sphinx-markdown-builder}

\lvliv{Usage}

Centroid give the possibility to store the point information with
\textbf{x} and \textbf{y} using a
\textbf{Point\textless double\textgreater{}} structure, and also the
\textbf{instensity} information.

It also contains a signal \textbf{newValue} that it could be emitted
when a new value is ready.

First you need to include the header file and use the namespace.

\begin{lstlisting}[language=c++, gobble=2]
  #include <PAT/Centroid.h>
  using namespace PAT;
\end{lstlisting}

Then you can define the variable ready to be used.

\begin{lstlisting}[language=c++, gobble=2]
  Centroid centroid(10, 20, 100);
  centroid.x = 10;
  centroid.y = 20;
  centroid.instensity = 100;
\end{lstlisting}

If you need to emit the signal use this function.

\begin{lstlisting}[language=c++, gobble=2]
  emit centroid.newValue();
\end{lstlisting}

\lvliv{UML}
\img{centroid}{Centroid UML.}

    \lvliii{KPA101}

This project create and interface between sensors, controllers,
actuators and all the functionalities of the KPA101.

\textbf{Requirements}

\includegraphics[scale=0.7]{img/shilds/cpp.png}
\includegraphics[scale=0.7]{img/shilds/cmake.png}
\includegraphics[scale=0.7]{img/shilds/git.png}
\includegraphics[scale=0.7]{img/shilds/doxygen.png}
\includegraphics[scale=0.7]{img/shilds/sphinx.png}
\includegraphics[scale=0.7]{img/shilds/win.png}

\lvlv{Generality}

\lvliv{Import}

Import as an external library into your project by copy-paste the
following lines in your \texttt{config.json}.

\begin{lstlisting}[language=javascript, gobble=2]
  {
    "name"     : "PATKPA101",
    "path"     : "gitlab.dei.unipd.it/PAT/KPA101.git",
    "tag"      : "HEAD",
    "available": "YES",
    "getGui"   : "NO"
  }
\end{lstlisting}

\lvliv{Prerequisites}

The following libraries are auto fetched from the gitlab.dei.unipd.it
host (ask the owner of this repo to become a member):

\begin{itemize}
  \tightlist
  \item
        \href{https://gitlab.dei.unipd.it/PAT/Utils.git}{Utils} 1.0.0
\end{itemize}

These other libraries need to be installed manually in your system:

\begin{itemize}
  \tightlist
  \item
        \href{https://www.qt.io/}{Qt} 5.14.2
\end{itemize}

The library documentation is generated through
\href{http://www.doxygen.nl/download.html}{Doxygen 1.8.13}. Additional
documentation in the \texttt{index} folder is generated through the
\href{https://www.anaconda.com/products/individual}{python3} package
\href{https://www.sphinx-doc.org/en/master/}{Sphinx} using the following
extensions (which you can install through pip3):

\begin{itemize}
  \tightlist
  \item
        \href{https://pypi.org/project/Sphinx/}{Sphinx 3.0.2}
  \item
        \href{https://sphinx-rtd-theme.readthedocs.io/en/stable/}{Sphinx read
          the doc theme} (to use the read the doc theme for html documentation)
  \item
        \href{https://pypi.org/project/breathe/}{Breathe} (to use the xml
        output of doxygen)
  \item
        \href{https://pypi.org/project/sphinx-markdown-builder/}{Sphinx-markdown-builder}
        (to generate the markdown version for gitlab wiki)
\end{itemize}

\texttt{pip3\ install\ Sphinx\ sphinx\_rtd\_theme\ breathe\ sphinx-markdown-builder}

\lvlv{Usage}

\lvliv{InputKPA}

InputKPA is a utility structure that contains the three voltagies that
KPA101 takes in input: \textbf{x}, \textbf{y}, \textbf{sum}.

First you need to include the header file and use the namespace.

\begin{lstlisting}[language=c++, gobble=2]
  #include <PAT/InputKPA.h>
  using namespace PAT;  
\end{lstlisting}

Then you can define the variable ready to be used.

\begin{lstlisting}[language=c++, gobble=2]
  InputKPA parameters;
  parameters.x = 1;
  parameters.y = 2;
  parameters.sum = 3;
\end{lstlisting}

\lvliv{KPA101}

KPA101 is a wrapper for the Kinesis libs that enhance their
functionalities.

First you need to include the header file and use the namespace.

\begin{lstlisting}[language=c++, gobble=2]
  #include <PAT/KPA101.h>
  using namespace PAT;
\end{lstlisting}

In order to define a variable you can choose to pass the \textbf{kpa id}
or to let it automatically search it. If the search fails the program
throw an exception "not found". To search the id you can use
\textbf{find} and \textbf{findAll} functions.

\begin{lstlisting}[language=c++, gobble=2]
  KPA101 kpa101;          // auto
  KPA101 kpa101("kpaid"); // with id
  KPA101 kpa101(KPA101::find());       // with find
  KPA101 kpa101(KPA101::findAll()[0]); // with findAll
\end{lstlisting}

\textbf{Sensor}

Here are listed the functions useful for sensors.

If you need the get the current input state you can use this function.

\begin{lstlisting}[language=c++, gobble=2]
  InputKPA getInputVoltage() const;
\end{lstlisting}

\textbf{Actuator}

Here are listed the functions useful for actuators.

You can set the output state of the KPA using one of these three
functions. Output voltages are between 10 and 0.

\begin{itemize}
  \tightlist
  \item
        The function \textbf{setOutputVoltage} set the output voltages at the
        selected values.
  \item
        The function \textbf{setOutputImpulse} send a single impulse with
        length \textbf{\_time\_ms}.
  \item
        The function \textbf{setOutputCombinedImpulse} send two impulses at
        the same time with with their respective lengths. I one finish before
        the other its voltage go to zero.
\end{itemize}

\begin{lstlisting}[language=c++, gobble=2]
  double voltageX = 1;
  double voltageY = 1;
  kpa101.setOutputVoltage(voltageX, voltageY);

  double voltagePeakX = 1;
  double voltagePeakY = 1;
  double time_ms      = 1;
  kpa101.setOutputImpulse(voltagePeakX, voltagePeakY, time_ms);

  double voltagePeakX = 1;
  double timeX_ms     = 1;
  double voltagePeakY = 1;
  double timeY_ms     = 1;
  kpa101.setOutputCombinedImpulse(voltagePeakX, timeX_ms, voltagePeakY, timeY_ms);
\end{lstlisting}

\textbf{Controller}

Here are listed the functions useful to use the PID controller
integrated in the KPA.

You can set the PID parameters using these functions. PID values are
between 1 and 0.

\begin{lstlisting}[language=c++, gobble=2]
  auto parameters = kpa101.getPIDParameters();
  parameters.value.proportionalGain = 1;
  parameters.value.integralGain = 0.001;
  parameters.value.differentialGain = 0.01;
  kpa101.setPIDParameters(parameters);
\end{lstlisting}

You can enable/disable or check if enabled the PID controller using
these functions.

\begin{lstlisting}[language=c++, gobble=2]
  if(!kpa101.isPIDActive())
    kpa101.enablePID(true);
  else
    kpa101.enablePID(false);
\end{lstlisting}

You can enable/disable the auto open closed loop. If the intensity of
the light is outside the interval \textbf{autoOpenInterval} then the PID
controller is stopped. The intensity value is bounded between 10 and 0.

\begin{lstlisting}[language=c++, gobble=2]
  auto interval = kpa101.getAutoOpenInterval();
  interval.value.min = 0.1;
  interval.value.max = 1.3;
  kpa101.setAutoOpenInterval(interval);
  if(!kpa101.getAutoOpenCloseLoop())
    kpa101.setAutoOpenCloseLoop(true);
\end{lstlisting}

The output of the PID controller is multiplied by a feedback constant
that range between -10 to +10 volts. Using this function you can set
it.

\begin{lstlisting}[language=c++, gobble=2]
  auto feedback = kpa101.getFeedBackGain();
  gain.value.x = 10;
  gain.value.y = 10;
  kpa101.setFeedBackGain(feedback);
\end{lstlisting}

\lvlv{UML}

\img{KPA101}{KPA101 UML.}
    \lvliii{Mirror}

This project contains the interface \textbf{Mirror} for all mirrors. The
actual implementations are: STT254.

\textbf{Requirements}

\includegraphics[scale=0.7]{img/shilds/cpp.png}
\includegraphics[scale=0.7]{img/shilds/cmake.png}
\includegraphics[scale=0.7]{img/shilds/git.png}
\includegraphics[scale=0.7]{img/shilds/doxygen.png}
\includegraphics[scale=0.7]{img/shilds/sphinx.png}
\includegraphics[scale=0.7]{img/shilds/win.png}

\lvliv{Generality}

\lvlv{Import}

Import as an external library into your project by copy-paste the
following lines in your \texttt{config.json}.

\begin{lstlisting}[language=javascript, gobble=2]
  {
    "name"     : "PATMirror",
    "path"     : "gitlab.dei.unipd.it/PAT/Mirror.git",
    "tag"      : "HEAD",
    "available": "YES",
    "getGui"   : "NO"
  }
\end{lstlisting}

\lvlv{Prerequisites}

The following libraries are auto fetched from the gitlab.dei.unipd.it
host (ask the owner of this repo to become a member):

\begin{itemize}
  \tightlist
  \item
        \href{https://gitlab.dei.unipd.it/PAT/KPA101.git}{KPA101} 1.0.0
\end{itemize}

These other libraries need to be installed manually in your system:

\begin{itemize}
  \tightlist
  \item
        \href{https://www.qt.io/}{Qt} 5.14.2
\end{itemize}

The library documentation is generated through
\href{http://www.doxygen.nl/download.html}{Doxygen 1.8.13}. Additional
documentation in the \texttt{index} folder is generated through the
\href{https://www.anaconda.com/products/individual}{python3} package
\href{https://www.sphinx-doc.org/en/master/}{Sphinx} using the following
extensions (which you can install through pip3):

\begin{itemize}
  \tightlist
  \item
        \href{https://pypi.org/project/Sphinx/}{Sphinx 3.0.2}
  \item
        \href{https://sphinx-rtd-theme.readthedocs.io/en/stable/}{Sphinx read
          the doc theme} (to use the read the doc theme for html documentation)
  \item
        \href{https://pypi.org/project/breathe/}{Breathe} (to use the xml
        output of doxygen)
  \item
        \href{https://pypi.org/project/sphinx-markdown-builder/}{Sphinx-markdown-builder}
        (to generate the markdown version for gitlab wiki)
\end{itemize}

\texttt{pip3\ install\ Sphinx\ sphinx\_rtd\_theme\ breathe\ sphinx-markdown-builder}

\lvliv{Usage}

\lvlv{Mirror}

The \textbf{Mirror} class is a interface for all mirrors. Here is the
include code.

\begin{lstlisting}[language=c++, gobble=2]
  #include <PAT/Mirror/Mirror.h>
\end{lstlisting}

To inherit the class you have to overwrite the two methods:
\textbf{setOutputImpulse}, \textbf{setOutputCombinedImpulse}. When the
impulse finish remember to emit a \textbf{movementCompleted} signal that
is already defined in \textbf{Mirror} class.

\begin{lstlisting}[language=c++, gobble=2]
  class YourClass : public Mirror {
    public:
      void setOutputImpulse(double _voltagePeakX, double _voltagePeakY, double _time_ms) override;
      void setOutputCombinedImpulse(double _voltagePeakX, double _timeX_ms, double _voltagePeakY, double _timeY_ms) override;
    };
\end{lstlisting}

\lvlv{STT254}

The \textbf{STT254} class is a Mirror implementation that it works with
the \textbf{KPA101}. Here is the include code.

\begin{lstlisting}[language=c++, gobble=2]
  #include <PAT/KPA101.h>
  #include <PAT/Mirror/STT254.h>
  using namespace PAT::Mirror;
\end{lstlisting}

When you istantiate the \textbf{STT254} you must pass a \textbf{KPA101}
istance. The \textbf{STT254} is typically used as polymorphism with
\textbf{Mirror}.

\begin{lstlisting}[language=c++, gobble=2]
  std::shared_ptr<PAT::KPA101> kpa101         = std::make_shared<PAT::KPA101>();
  std::shared_ptr<PAT::Mirror::Mirror> mirror = std::make_shared<PAT::Mirror::STT254>(kpa101);
\end{lstlisting}

You can set the output state of the \textbf{Mirror} using one of these
two functions. Output voltages are between 10 and 0. When the functions
finish they emit a a \textbf{movementCompleted} signal.

\begin{itemize}
  \tightlist
  \item
        The function \textbf{setOutputImpulse} send a single impulse with
        length \textbf{\_time\_ms}.
  \item
        The function \textbf{setOutputCombinedImpulse} send two impulses at
        the same time with with their respective lengths. I one finish before
        the other its voltage go to zero.
\end{itemize}

\begin{lstlisting}[language=c++, gobble=2]
  double voltagePeakX = 1;
  double voltagePeakY = 1;
  double time_ms      = 1;
  mirror->setOutputImpulse(voltagePeakX, voltagePeakY, time_ms);
  
  double voltagePeakX = 1;
  double timeX_ms     = 1;
  double voltagePeakY = 1;
  double timeY_ms     = 1;
  mirror->setOutputCombinedImpulse(voltagePeakX, timeX_ms, voltagePeakY, timeY_ms);
\end{lstlisting}

\lvliv{UML}

\img{Mirror}{Mirror UML.}

    \lvliii{PSD}

This project contains the interface \textbf{PSD} for all PSD. The actual
implementations are: PDP90A.

\textbf{Requirements}

\includegraphics[scale=0.7]{img/shilds/cpp.png}
\includegraphics[scale=0.7]{img/shilds/cmake.png}
\includegraphics[scale=0.7]{img/shilds/git.png}
\includegraphics[scale=0.7]{img/shilds/doxygen.png}
\includegraphics[scale=0.7]{img/shilds/sphinx.png}
\includegraphics[scale=0.7]{img/shilds/win.png}

\lvliv{Generality}

\lvlv{Import}

Import as an external library into your project by copy-paste the
following lines in your \texttt{config.json}.

\begin{lstlisting}[language=javascript, gobble=2]
  {
    "name"     : "PATPSD",
    "path"     : "gitlab.dei.unipd.it/PAT/PSD.git",
    "tag"      : "HEAD",
    "available": "YES",
    "getGui"   : "YES"
  }
\end{lstlisting}

\lvlv{Prerequisites}

The following libraries are auto fetched from the gitlab.dei.unipd.it
host (ask the owner of this repo to become a member):

\begin{itemize}
  \tightlist
  \item
        \href{https://gitlab.dei.unipd.it/PAT/KPA101.git}{KPA101} 1.0.0
  \item
        \href{https://gitlab.dei.unipd.it/PAT/Centroid.git}{Centroid} 1.0.0
\end{itemize}

These other libraries need to be installed manually in your system:

\begin{itemize}
  \tightlist
  \item
        \href{https://www.qt.io/}{Qt} 5.14.2
\end{itemize}

The library documentation is generated through
\href{http://www.doxygen.nl/download.html}{Doxygen 1.8.13}. Additional
documentation in the \texttt{index} folder is generated through the
\href{https://www.anaconda.com/products/individual}{python3} package
\href{https://www.sphinx-doc.org/en/master/}{Sphinx} using the following
extensions (which you can install through pip3):

\begin{itemize}
  \tightlist
  \item
        \href{https://pypi.org/project/Sphinx/}{Sphinx 3.0.2}
  \item
        \href{https://sphinx-rtd-theme.readthedocs.io/en/stable/}{Sphinx read
          the doc theme} (to use the read the doc theme for html documentation)
  \item
        \href{https://pypi.org/project/breathe/}{Breathe} (to use the xml
        output of doxygen)
  \item
        \href{https://pypi.org/project/sphinx-markdown-builder/}{Sphinx-markdown-builder}
        (to generate the markdown version for gitlab wiki)
\end{itemize}

\texttt{pip3\ install\ Sphinx\ sphinx\_rtd\_theme\ breathe\ sphinx-markdown-builder}

\lvliv{Usage}

\lvlv{PSD}

The \textbf{PSD} class is a interface for all PSD. Here is the include
code.

\begin{lstlisting}[language=c++, gobble=2]
  #include <PAT/PSD/PSD.h>
\end{lstlisting}

To inherit the class you have to overwrite the three methods:
\textbf{start}, \textbf{stop}, \textbf{getIntensity}.

\begin{lstlisting}[language=c++, gobble=2]
  class YourClass : public PSD {
    public:
      void start() override;
      void stop() override;
      double getIntensity() const override;
    };    
\end{lstlisting}

\lvlv{PDP90A}

The \textbf{PDP90A} class is a PSD implementation that it works with the
\textbf{KPA101}. Here is the include code.

\begin{lstlisting}[language=c++, gobble=2]
  #include <PAT/KPA101.h>
  #include <PAT/PSD/PDP90A.h>
  using namespace PAT::PSD;  
\end{lstlisting}

When you istantiate the \textbf{PDP90A} you must pass a \textbf{KPA101}
istance. The \textbf{PDP90A} is typically used as polymorphism with
\textbf{PSD}.

\begin{lstlisting}[language=c++, gobble=2]
  std::shared_ptr<PAT::KPA101> kpa101   = std::make_shared<PAT::KPA101>();
  std::shared_ptr<PAT::PSD::PSD> psd = std::make_shared<PAT::PSD::PDP90A>(kpa101);
\end{lstlisting}

You can enable/disable or check if enabled the PDP90A's acquisition
using these functions.

\begin{lstlisting}[language=c++, gobble=2]
  if(!psd->isActive())
    psd->start();
 else
    psd->stop();
\end{lstlisting}

You can set the frame rate of the acquisition using these functions. At
every frame the PSD calculates the centroid and then it emit a signal
\textbf{newFrameSignal}.

\begin{lstlisting}[language=c++, gobble=2]
  auto framerate  = psd->getFrameRate();
  framerate.value = 30;
  psd->setFrameRate(framerate);
\end{lstlisting}

You can get the other parameters using these functions.

\begin{lstlisting}[language=c++, gobble=2]
  PAT::Utils::Point<double> sensorSize = psd->getSensorSize();
  std::shared_ptr<PAT::Utils::BoundedParameter<PAT::Utils::Circle<int>>> target = psd->getTarget();
  std::shared_ptr<PAT::Centroid> centroid = psd->getCentroid();
  PAT::Utils::Point<double> sensorSize = psd->getSensorSize();
  double intensity = psd->getIntensity();
\end{lstlisting}

\lvlv{PSDWiget}

The \textbf{PSDWiget} class is a widget fot the PSD's implementations.
Here I present the PDP90A example. The include code is the following.

\begin{lstlisting}[language=c++, gobble=2]
  #include <QApplication>
  #include <PAT/PSD/gui/PSDWidget.h>
  #include <PAT/PSD/PDP90A.h>
  using namespace PAT::PSD;
\end{lstlisting}

This is a simple \textbf{main} example.

\begin{lstlisting}[language=c++, gobble=2]
  int main(int argc, char *argv[]) {
    QApplication a(argc, argv);
    PSDWidget cameraWidget;
    std::shared_ptr<PAT::KPA101> kpa101 = std::make_shared<PAT::KPA101>();
    std::shared_ptr<PSD> camera         = std::make_shared<PDP90A>(kpa101);
    cameraWidget.setPSD(camera);
    cameraWidget.show();
    camera->start();
    return a.exec();
 }
\end{lstlisting}

If you want instead integrate the widget into another GUI (layout) you
can use this code.

\begin{lstlisting}[language=c++, gobble=2]
  QWidget *window                     = new QWidget();
  QVBoxLayout *mainLayout             = new QVBoxLayout();
  PSDWidget *psdWidget                = new PSDWidget();
  std::shared_ptr<PAT::KPA101> kpa101 = std::make_shared<PAT::KPA101>();
  std::shared_ptr<PSD> camera         = std::make_shared<PDP90A>(kpa101);
  
  psdWidget->setPSD(psd);
  mainLayout->addWidget((QWidget *)psdWidget);
  window->setLayout(mainLayout);
  window->show();
\end{lstlisting}

\lvliv{UML}

\img{PSD}{PSD UML.}

  }

\lvliii{Camera}
\img{Camera}{Camera UML.}

\lvliii{Controller}
\img{Controller}{Controller UML.}

\lvliii{Setup}
\img{Setup}{Simplified summary UML scheme.}