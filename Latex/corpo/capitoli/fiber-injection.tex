\lvli{Fiber injection setup}
Sebbene il software PAT si occupi di mantenere l'allineamento, per avere delle buone condizioni iniziali è necessario un allineamento manuale. La procedura di allineamento verrà riportata in questa sezione, per semplificarla verranno definiti degli acronimi delle principali componenti in Tab. \ref{table:1}.

\begin{table}[h!]
  \centering
  \begin{tabular}{ |c|c|c| }
    \hline
    Acronym & Description                                \\
    \hline
    AL      & Alice laser @1550nm used for communication \\
    ALF     & Alice laser @1545nm used for PAT fine      \\
    ALC     & Alice laser @1540nm used for PAT coarse    \\
    ACC     & Alice camera @1540nm used for PAT coarse   \\
    BL      & Bob laser's slot used for alignment        \\
    BF      & Bob fiber @1550nm used for communication   \\
    BP      & Bob PSD @1545nm used for PAT fine          \\
    BCC     & Bob camera @1540nm used for PAT coarse     \\
    BLC     & Bob laser @1540nm used for PAT coarse      \\
    FMS     & Bob fast-steering mirror                   \\
    DBS     & Bob dichroic beam splitter                 \\
    PSD     & Bob PSD                                    \\
    \hline
  \end{tabular}
  \caption{Table to test captions and labels}
  \label{table:1}
\end{table}

\lvlii{Coarse alignment}
\begin{itemize}
  \item montare dei laser visibili e collimati sulla montatura al posto di AL, BL
  \item far divergere i fasci dei laser ALC, BLC
  \item montare i laser ALC, BLC e assicurarsi che siano "paralleli" rispettivamente da AL, BL
  \item assicurarsi che le camere ACC, BCC mettano a fuoco sul piano dei laser ALC, BLC
  \item puntare Alice su Bob e vice versa, assicurarsi che ALC, BLC siano "abbastanza" centrati nelle camere BCC, ACC. In caso contrario correggere con il tip-tilt dei laser
  \item salvarsi i centri di allineamento delle camere
  \item fare una calibrazione con il software PAT su entrambe le montature
  \item abilitare il controllo One Shot
\end{itemize}

\lvlii{Fine alignment}
\begin{itemize}
  \item collimare AL, AF e montarli sulla montatura ed assicurarsi che siano paralleli (coassiali)
  \item montare il FSM centrato in x/y e a 45 gradi rispetto il percorso ottico
  \item montare il DBS centrato in x/y e a 45 gradi rispetto il percorso ottico
  \item montare la lente e il PSD in maniera che il PSD sia posizionato sul fuoco della lente
  \item accendere AL, ALF ed assicurarsi che arrivino centrati in BF, BP. In caso contrario correggere con il posizionamento di FSM, DBS
  \item fare una calibrazione con il software PAT sul PSD
  \item fare un Home con il software PAT per centrare il FSM, e aggiustare il suo posizionamento che sia visibile nel PSD
  \item ricalibrare con il software PAT
  \item attivare il controllo PID
\end{itemize}

\lvlii{Multi-mode fiber alignment}
% \begin{itemize}
%   \item fare un home in modo che sia in posizione neutra
%   \item sparare con il trasmettitore 1550+980 e vedere che non divergano
%         collimare il trasmettitore 1550+980 e vedere che siano collimato (togliere la fibra) (980 non lo sarà)
%   \item controllare che 1550 sia 3.6mm nel punto della fibra $(4 * 1550 * 10^-9 / \pi * 18.4 / 10^-5)$
%   \item controllare che il 980 sia centrato nella cage
%   \item vedere la distanza del fuoco del 980
%   \item mettere una lente con il coating B (300mm) e controllare ancora dove sia il fuoco (ci aspettiamo sia intorno al 300)
%   \item posizionare il PSD nella posizione del fuoco (se non abbiamo spazio muovere la lente)
%   \item smontare il laser e metterlo sul banco
%   \item controllare che non diverga
%   \item controllare che sia collimato
%   \item rimontate tutto ed attivare il controllo
%   \item provare ad accoppiare in fibra
%   \item sparare due laser da 1550 e vedere che uno entra e uno esce nel posto giusto
%   \item misurare le perdite in dB in uscita dal laser 1550 prima di entrata in fibra nel trasmettitore (max 30 dB)
% \end{itemize}
\lvlii{Single-mode fiber alignment}
