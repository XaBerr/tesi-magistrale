\lvli{Fiber injection setup}
Sebbene il software PAT si occupi di mantenere l'allineamento, per avere delle buone condizioni iniziali è necessario un allineamento manuale. La procedura di allineamento verrà riportata in questa sezione, per semplificarla verranno definiti degli acronimi delle principali componenti in Tab. \ref{table:1}.

\begin{table}[h!]
  \centering
  \begin{tabular}{ |c|c|c| }
    \hline
    Acronym & Description                                              \\
    \hline
    AL      & Alice laser @1550nm used for communication               \\
    ALF     & Alice laser @1545nm used for PAT fine                    \\
    ALC     & Alice laser @1540nm used for PAT coarse                  \\
    ACC     & Alice camera @1540nm used for PAT coarse                 \\
    \hline
    BL      & Bob laser's slot used for alignment                      \\
    BF      & Bob fiber-port collimator @1550nm used for communication \\
    BP      & Bob PSD @1545nm used for PAT fine                        \\
    BCC     & Bob camera @1540nm used for PAT coarse                   \\
    BLC     & Bob laser @1540nm used for PAT coarse                    \\
    FSM     & Bob fast-steering mirror                                 \\
    DBS     & Bob dichroic beam splitter                               \\
    PSD     & Bob PSD                                                  \\
    \hline
  \end{tabular}
  \caption{Table to test captions and labels}
  \label{table:1}
\end{table}

\lvlii{Coarse alignment}
\begin{itemize}
  \item montare dei laser visibili e collimati sulla montatura al posto di AL, BL
  \item far divergere i fasci dei laser ALC, BLC
  \item montare i laser ALC, BLC e assicurarsi che siano "paralleli" rispettivamente da AL, BL
  \item assicurarsi che le camere ACC, BCC mettano a fuoco sul piano dei laser ALC, BLC
  \item puntare Alice su Bob e vice versa, assicurarsi che ALC, BLC siano "abbastanza" centrati nelle camere BCC, ACC. In caso contrario correggere con il tip-tilt dei laser
  \item salvarsi i centri di allineamento delle camere
  \item fare una calibrazione con il software PAT su entrambe le montature
  \item abilitare il controllo One Shot
\end{itemize}

\lvlii{Fine alignment}
\begin{itemize}
  \item collimare AL, AF e montarli sulla montatura ed assicurarsi che siano paralleli (coassiali)
  \item montare il FSM centrato in x/y e a 45 gradi rispetto il percorso ottico
  \item montare il DBS centrato in x/y e a 45 gradi rispetto il percorso ottico
  \item montare la BF
  \item montare la lente e il PSD in maniera che il PSD sia posizionato sul fuoco della lente
  \item accendere AL, ALF ed assicurarsi che arrivino centrati in BF, BP. In caso contrario correggere con il posizionamento di FSM, DBS
  \item fare una calibrazione con il software PAT sul PSD
  \item fare un Home con il software PAT per centrare il FSM, e aggiustare il suo posizionamento che sia visibile nel PSD
  \item ricalibrare con il software PAT
  \item attivare il controllo PID
\end{itemize}

\lvlii{Multi-mode fiber alignment}
\begin{itemize}
  \item smontare il laser e metterlo sul banco
  \item smontare la BF e metterla sul manco dentro una cage attaccandoci una fibra multi modo
  \item collegari una fibra e sparare un laser visibile attraverso la fibra
  \item assicurarsi che il fascio uscente sia allineato con la cage e sia collimato
  \item controllare che AL abbia un diametro di 3.6mm dove verrà posizionata la BF $(4 * 1550 * 10^-9 / \pi * 18.4 / 10^-5)$. Misurare la potenza del laser in quel punto.
  \item rimontare BF con il laser visibile attaccato
  \item controllare che il laser AL e il laser uscente da BF siano allineati. In caso contrario correggere con il movimenti di precisione della montatura del DBS (tutti i sistemi PAT devono essere accesi)
  \item assicurarsi di avere un buon accoppiamento in fibra controllando che le perdite prima e dopo la fibra siano inferiori al 40\%
\end{itemize}

\lvlii{Single-mode fiber alignment}
\img{fiberport}{This image shows the adjustments sequence for the fiber-port. The image is taken in the Thorlabs website.}
\begin{itemize}
  \item smontare la fibra multi modo e montare quella a singolo modo
  \item Choose an sequence to make your adjustments, and keep to that sequence as shown in \fig{fiberport}.
  \item Turn each adjuster clockwise to maximize the output, then continue to turn slightly beyond local maxima (to ~95\% of your local maximum). If turning an adjuster clockwise decreases output, skip that adjuster for that round of adjustments and repeat. Once the local maxima values begin to decrease, reverse the direction, and turn each adjuster to maximize the output, and not beyond.
\end{itemize}
