\lvli{Fiber injection experiment}
In this experiment, a PAT system was developed between two telescopes in order to realize QKD with single-mode fiber injection.
The objective of this experiment was to send a signal between two telescopes, making sure that it was injected in single-mode fibre, with attenuation lower than the 50\%.

Although the PATSW maintains alignment, manual alignment of the single components of the PATHW is required. The alignment procedure will be reported in this section, to simplify it will be defined acronyms of the main components in Tab. \ref{table:1}.

\begin{table}[h!]
  \centering
  \begin{tabular}{ |c|c|c| }
    \hline
    Acronym & Description                                              \\
    \hline
    AL      & Alice laser @1550nm used for communication               \\
    ALF     & Alice laser @1545nm used for PAT fine                    \\
    ALC     & Alice laser @1540nm used for PAT coarse                  \\
    ACC     & Alice camera @1540nm used for PAT coarse                 \\
    \hline
    BL      & Bob laser's slot used for alignment                      \\
    BF      & Bob fiber-port collimator @1550nm used for communication \\
    BP      & Bob PSD @1545nm used for PAT fine                        \\
    BCC     & Bob camera @1540nm used for PAT coarse                   \\
    BLC     & Bob laser @1540nm used for PAT coarse                    \\
    FSM     & Bob fast-steering mirror                                 \\
    DBS     & Bob dichroic beam splitter                               \\
    PSD     & Bob PSD                                                  \\
    \hline
  \end{tabular}
  \caption{Table to test captions and labels}
  \label{table:1}
\end{table}
The complete alignment procedure for PATHW will be divided into parts. At the end of each part it will be needed an activation of the corresponding PATSW in order to move into the next one.

\lvlii{Coarse alignment}
\begin{itemize}
  \item mount visible and collimated lasers on the frame instead of AL, BL
  \item diverge the laser beams ALC, BLC
  \item mount ALC, BLC lasers and ensure that they are "parallel" to AL, BL, respectively
  \item ensure that ACC, BCC cameras focus on the plane of ALC, BLC lasers
  \item bet Alice on Bob and vice versa, make sure ALC, BLC are "quite" centered in the BCC, ACC rooms. Otherwise correct with laser tip-tilt
  \item save the camera alignment points
  \item make a calibration with PATSW on both mounts
  \item enable the One shot controller
\end{itemize}

\lvlii{Fine alignment}
\begin{itemize}
  \item collimate AL, AF and mount them on the mount and make sure they are parallel (coaxial)
  \item mount the FSM centered in x/y and 45 degrees to the optical path
  \item mount the DBS centered in x/y and 45 degrees to the optical path
  \item mount the BF
  \item mount the lens and PSD so that the PSD is positioned on the lens focus
  \item turn on AL, ALF and make sure they arrive centered in BF, BP. Otherwise correct with the placement of FSM, DBS
  \item calibration with PATSW on PSD
  \item make a Home with PATSW to put the FSM in its tip-tilt neutral position, then correct its placement so the laser is still visible in the PSD
  \item recalibrate with PATSW
  \item activate the PID control
\end{itemize}

\lvlii{Multi-mode fiber alignment}
\begin{itemize}
  \item disassemble the laser and put it on the table
  \item disassemble the BF and put it on the table inside a cage attaching a multi-mode fiber
  \item shoot a visible laser through the fiber and ensure that the outgoing beam is aligned with the cage and is collimated
  \item check that AL has a diameter of 3.6mm where the BF will be placed $(4 * 1550 * 10^-9 / \pi * 18.4 / 10^-5)$. Measure the laser power at that point
  \item reassemble BF in the mount with visible laser attached
  \item check that the AL laser and the BF laser are aligned. Otherwise correct with the precision movements of the DBS frame (all PAT systems must be switched on)
  \item make sure to have a good fiber coupling by checking that the losses before and after the fiber are less than 40\%
\end{itemize}

\lvlii{Single-mode fiber alignment}
\img{fiberport}{This image shows the adjustments sequence for the fiber-port. The image is taken in the Thorlabs website.}
\begin{itemize}
  \item disassemble the multi-mode fiber and mount the single-mode fiber
  \item choose an sequence to make your adjustments, and keep to that sequence as shown in \fig{fiberport}
  \item turn each adjuster clockwise to maximize the output, then continue to turn slightly beyond local maxima (to ~95\% of your local maximum). If turning an adjuster clockwise decreases output, skip that adjuster for that round of adjustments and repeat. Once the local maxima values begin to decrease, reverse the direction, and turn each adjuster to maximize the output, and not beyond
  \item make sure to have a good fiber coupling by checking that the losses before and after the fiber are less than 50\%
\end{itemize}
