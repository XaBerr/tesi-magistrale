\lvli{QKD software design}
\lvlii{Memorization, standardization and expansion of the QKD software}
All of the code that control the QKD was done temporarily on several computers without a particular organization. It was therefore necessary to implement an easy system that would allow standardization, memorization, and simplification of code expansion.

To store the code, it was chosen to use GitLab, a web-platform that allows the storage and versioning of the code via Git. The code can then be stored in repositories that are organized in groups.

To create a standardization has been chosen to create a \idx{Boilerplate}, a ready-made code that serves as a template for the realization of various projects. The Boilerplate was built in c++ with the integration of Cmake and Qt.
\idx{CMake} is a system that simplifies the compilation of sources, while \idx{Qt} is a set of libraries that simplify the graphical interface for the c++. The Boilerplate also contains two types of documentation: README and Sphinx.
\idx{README} is a reStructuredText (.rst) file in which a template allows you to quickly describe how to use the library.
The documentation in \idx{Sphinx} is more complex, it uses a python program that collects all the inline comments in the code and provides the class structure visible through web pages.

To simplify code expansion, it was decided to create a CMake-based packet manager and integrate it directly into the Boilerplate. This new system allows to list in a file "config.json" the dependencies of every project, that are the libraries necessary for its compilation.
When the project is compiled, dependencies are automatically downloaded and compiled.

\lvlii{Boilerplate documentation}
Boilerplate documentation is stored in a separate repository called BoilerplateQt.wiki. It is made in markdown (.md). This section will list its main sections.
  {\parindent0pt
    \lvliii{Setup your environment}
\lvliv{Prerequisites}

These libraries need to be installed manually in your system:

\begin{itemize}
      \tightlist
      \item
            \href{https://www.qt.io/}{Qt 5.14.2}
      \item
            \href{http://www.doxygen.nl/download.html}{Doxygen 1.8.13}
      \item
            \href{https://www.anaconda.com/products/individual}{python3}
\end{itemize}

This can be installed through \textbf{pip3} using a single command line:

\begin{itemize}
      \tightlist
      \item
            \href{https://pypi.org/project/Sphinx/}{Sphinx 3.0.3}
      \item
            \href{https://sphinx-rtd-theme.readthedocs.io/en/stable/}{Sphinx read
                  the doc theme 0.4.3} (to use the read the doc theme for html
            documentation)
      \item
            \href{https://pypi.org/project/breathe/}{Breathe 4.16.0} (to use the
            xml output of doxygen)
      \item
            \href{https://pypi.org/project/sphinx-markdown-builder/}{Sphinx-markdown-builder
                  0.5.4} (to generate the markdown version for gitlab wiki)
\end{itemize}

\texttt{pip3\ install\ Sphinx\ sphinx\_rtd\_theme\ breathe\ sphinx-markdown-builder}

On \textbf{Windows} add the path of the \texttt{../Anaconda3/Scripts} folder to the system variable \textbf{PATH}.

\lvliv{Setup procedure}

\begin{itemize}
      \tightlist
      \item
            Create a new project on gitlab \{YourProject\} (the name of the
            project must be \href{https://en.wikipedia.org/wiki/Camel_case}{upper
                  camel case}).
      \item
            Download \{YourProject\} on the computer.
      \item
            Download this repository

            \texttt{git\ clone\ https://gitlab.dei.unipd.it/Calderaro/BoilerplateQT.git}.
      \item
            Copy everything except the folder \textbf{.git} from this repository
            to \{YourProject\} folder.
      \item
            (optional for QtCreator) Configure the project as described in the
            Installation.
      \item
            Replace every string \textbf{BoilerplateQt} with \{YourProject\} in
            every file (Recommended to use QtCreator advanced replace).
      \item
            Replace every string \textbf{boilerplateQt} with \{yourProject\} in
            every file (Recommended to use QtCreator advanced replace).
      \item
            Replace every string \textbf{boilerplateqt} with \{yourproject\} in
            every file (Recommended to use QtCreator advanced replace).
      \item
            Replace every string \textbf{BOILERPLATEQT} with \{YOURPROJECT\} in
            every file (Recommended to use QtCreator advanced replace).
      \item
            Rename every file that contains \textbf{boilerplateqt} with
            \{yourproject\}:
            \begin{itemize}
                  \tightlist
                  \item
                        `/gui/boilerplateqtwidget.cpp` to `/gui/{yourproject}widget.cpp`.

                  \item
                        `/gui/boilerplateqtwidget.ui` to `/gui/{yourproject}widget.ui`.

                  \item
                        `/src/boilerplateqt.cpp` to `/src/{yourproject}.cpp`.

                  \item
                        `/tests/boilerplateqttest.cpp` to `/tests/{yourproject}test.cpp`.

                  \item
                        `/include/BoilerplateQt/boilerplateqt.h` to `/include/BoilerplateQt/{yourproject}.h`.
            \end{itemize}
\end{itemize}


\begin{itemize}
      \tightlist
      \item
            Rename the folder \texttt{/include/BoilerplateQt} with
            \texttt{/include/\{YourProject\}}.
      \item
            Open \texttt{config.json} and substitute with your data:
      \item \textbf{name}: name of the project {YourProject}.

      \item \textbf{version}: last tag in {YourProject} master history.

      \item \textbf{description}: short description of {YourProject}.

      \item \textbf{GUI}: whether {YourProject} have a GUI. \textbf{If you do not have a gui remove} `/gui` \textbf{and} `/include/{YourProject}/gui` \textbf{folders}.

      \item \textbf{modules}: external libraries to be fetched from an online repositoty.
      \item
            (Optional for QtCreator):
            \begin{itemize}
                  \tightlist
                  \item Close the project on QtCreator.

                  \item Remove the file `CMakeLists.txt.user` and build directory (if any).

                  \item Configure the project as described in the Installation.
            \end{itemize}
\end{itemize}


If you correctly executed the above steps the project should build with
no problems. From this point on you can start:

\begin{itemize}
      \tightlist
      \item
            adding new source files and/or change the existing ones.
      \item
            adding and removing libraries dependeces.
\end{itemize}

    \lvlv{Adding new classes to the src folder}

In the \texttt{src/CMakeLists.txt} add the source and header (only the
headers not exported in the library) files in the \textbf{add\_library}
function. If you want the class to be exported in the library, so that
an external program can use it, put the header in the
\texttt{include/ProjName} folder.


\lvlv{Adding new classes to gui folder (should not be necessary
  but why not)}

In the \texttt{gui/CMakeLists.txt} add the source and header (only the
headers not exported in the library) files in the \textbf{add\_library}
function. If you want the class to be exported in the library, so that
an external program can use it, put the header in the
\texttt{include/ProjName/gui} folder.


\lvlv{Adding new executables to the apps folder}

In the \texttt{apps/CMakeLists.txt} add the \textbf{add\_executable}
function to create a new target executable:

\begin{lstlisting}[language=c++]
  add_executable(
  "${PROJECT_NAME}NewApp"
  newapp.cpp
  )
\end{lstlisting}

Also add the libraries that should be linked to the ``NewApp'' with
target\emph{link}libraries:

\begin{lstlisting}[language=c++]
  target_link_libraries(
    "${PROJECT_NAME}NewApp"
    PRIVATE Qt5::Gui Qt5::Widgets "${PROJECT_NAME}Gui" ${PROJECT_NAME} ${SUBMODULES_NAME}
  )
\end{lstlisting}

    \lvliv{Configuring your root CMakeLists.txt}

This project builds two libraries called BoilerplateQt and
BoilerplateQtGui. To import these libraries in your personal project you
can do the following. Assuming your project structure is based on
BoilerplateQt, you can add to the \texttt{config.json} the following
lines:

\begin{lstlisting}[language=javascript]
"modules": [
  {
    "other module": "info"
  },
  {
    "name": "BoilerplateQt",
    "path": "gitlab.dei.unipd.it/Calderaro/BoilerplateQt.git",
    "tag": "1.0.0",
    "avaiable": "YES",
    "getGui": "YES"
  }
]
\end{lstlisting}

\textbf{Keep in mind that this will force your submodules to use this version (the one described in the ``tag'') of BoilerplateQt. Make sure that your submodule works fine with the version you set.}

This will add two library targets: BoilerplateQt and BoilerplateQtGui.
Other options:

\begin{itemize}
  \item
        If you do not want to import the BoilerplateQtGui library set
        \texttt{"getGui":\ "NO"}
\end{itemize}

\begin{itemize}
  \item
        You can just force your submodules to use a particular version of the
        BoilerplateQt libraries by setting \texttt{"available":\ "NO"}. Your
        project will not import the BoilerplateQt and BoilerplateQtGui
        targets.
\end{itemize}

\lvliv{Add the BoilerplateQtWidget to your GUI
  Widget}

Show the BoilerplateQtWidget inside your GUI Widget following these
steps:

\begin{itemize}
  \item
        create an empty Widget in your GUI Widget and name it as
        ``boilerplateQtWidget''.
\end{itemize}

\begin{itemize}
  \item
        right-click on the boilerplateQtWidget and select ``promote to''.
\end{itemize}

\begin{itemize}
  \item
        Use ``QWidget'' as base class.
\end{itemize}

\begin{itemize}
  \item
        In ``Promoted class name'' insert ``BoilerplateQtWidget''.
\end{itemize}

\begin{itemize}
  \item
        In ``Header file'' insert
        \textless BoilerplateQt/gui/boilerplateqtwidget.h\textgreater.
\end{itemize}

Connect BoilerplateQt business logic to the presentation of
BoilerplateQt Widget following these steps:

\begin{itemize}
  \item
        add two private pointers of type BoilerplateQt and SimpleLabel to your
        GUI Widget class.
\end{itemize}

\begin{lstlisting}[language=c++]
BoilerplateQt *boilerplateQt;
SimpleLabel *simpleLabel;
\end{lstlisting}

\begin{itemize}
  \item
        assign two objects to the pointers in a method of your GUI Widget
        class.
\end{itemize}

\begin{itemize}
  \item
        call the setBoilerplateQt method of BoilerplateQtWidget, passing the
        two pointers as argument:
\end{itemize}

\begin{lstlisting}[language=c++]
  ui->boilerplateQtWidget->setBoilerplateQt(boilerplateQt, simpleLabel).
\end{lstlisting}

    % Options for packages loaded elsewhere
\PassOptionsToPackage{unicode}{hyperref}
\PassOptionsToPackage{hyphens}{url}
%
\documentclass[
]{article}
\usepackage{lmodern}
\usepackage{amssymb,amsmath}
\usepackage{ifxetex,ifluatex}
\ifnum 0\ifxetex 1\fi\ifluatex 1\fi=0 % if pdftex
  \usepackage[T1]{fontenc}
  \usepackage[utf8]{inputenc}
  \usepackage{textcomp} % provide euro and other symbols
\else % if luatex or xetex
  \usepackage{unicode-math}
  \defaultfontfeatures{Scale=MatchLowercase}
  \defaultfontfeatures[\rmfamily]{Ligatures=TeX,Scale=1}
\fi
% Use upquote if available, for straight quotes in verbatim environments
\IfFileExists{upquote.sty}{\usepackage{upquote}}{}
\IfFileExists{microtype.sty}{% use microtype if available
  \usepackage[]{microtype}
  \UseMicrotypeSet[protrusion]{basicmath} % disable protrusion for tt fonts
}{}
\makeatletter
\@ifundefined{KOMAClassName}{% if non-KOMA class
  \IfFileExists{parskip.sty}{%
    \usepackage{parskip}
  }{% else
    \setlength{\parindent}{0pt}
    \setlength{\parskip}{6pt plus 2pt minus 1pt}}
}{% if KOMA class
  \KOMAoptions{parskip=half}}
\makeatother
\usepackage{xcolor}
\IfFileExists{xurl.sty}{\usepackage{xurl}}{} % add URL line breaks if available
\IfFileExists{bookmark.sty}{\usepackage{bookmark}}{\usepackage{hyperref}}
\hypersetup{
  hidelinks,
  pdfcreator={LaTeX via pandoc}}
\urlstyle{same} % disable monospaced font for URLs
\usepackage{color}
\usepackage{fancyvrb}
\newcommand{\VerbBar}{|}
\newcommand{\VERB}{\Verb[commandchars=\\\{\}]}
\DefineVerbatimEnvironment{Highlighting}{Verbatim}{commandchars=\\\{\}}
% Add ',fontsize=\small' for more characters per line
\newenvironment{Shaded}{}{}
\newcommand{\AlertTok}[1]{\textcolor[rgb]{1.00,0.00,0.00}{\textbf{#1}}}
\newcommand{\AnnotationTok}[1]{\textcolor[rgb]{0.38,0.63,0.69}{\textbf{\textit{#1}}}}
\newcommand{\AttributeTok}[1]{\textcolor[rgb]{0.49,0.56,0.16}{#1}}
\newcommand{\BaseNTok}[1]{\textcolor[rgb]{0.25,0.63,0.44}{#1}}
\newcommand{\BuiltInTok}[1]{#1}
\newcommand{\CharTok}[1]{\textcolor[rgb]{0.25,0.44,0.63}{#1}}
\newcommand{\CommentTok}[1]{\textcolor[rgb]{0.38,0.63,0.69}{\textit{#1}}}
\newcommand{\CommentVarTok}[1]{\textcolor[rgb]{0.38,0.63,0.69}{\textbf{\textit{#1}}}}
\newcommand{\ConstantTok}[1]{\textcolor[rgb]{0.53,0.00,0.00}{#1}}
\newcommand{\ControlFlowTok}[1]{\textcolor[rgb]{0.00,0.44,0.13}{\textbf{#1}}}
\newcommand{\DataTypeTok}[1]{\textcolor[rgb]{0.56,0.13,0.00}{#1}}
\newcommand{\DecValTok}[1]{\textcolor[rgb]{0.25,0.63,0.44}{#1}}
\newcommand{\DocumentationTok}[1]{\textcolor[rgb]{0.73,0.13,0.13}{\textit{#1}}}
\newcommand{\ErrorTok}[1]{\textcolor[rgb]{1.00,0.00,0.00}{\textbf{#1}}}
\newcommand{\ExtensionTok}[1]{#1}
\newcommand{\FloatTok}[1]{\textcolor[rgb]{0.25,0.63,0.44}{#1}}
\newcommand{\FunctionTok}[1]{\textcolor[rgb]{0.02,0.16,0.49}{#1}}
\newcommand{\ImportTok}[1]{#1}
\newcommand{\InformationTok}[1]{\textcolor[rgb]{0.38,0.63,0.69}{\textbf{\textit{#1}}}}
\newcommand{\KeywordTok}[1]{\textcolor[rgb]{0.00,0.44,0.13}{\textbf{#1}}}
\newcommand{\NormalTok}[1]{#1}
\newcommand{\OperatorTok}[1]{\textcolor[rgb]{0.40,0.40,0.40}{#1}}
\newcommand{\OtherTok}[1]{\textcolor[rgb]{0.00,0.44,0.13}{#1}}
\newcommand{\PreprocessorTok}[1]{\textcolor[rgb]{0.74,0.48,0.00}{#1}}
\newcommand{\RegionMarkerTok}[1]{#1}
\newcommand{\SpecialCharTok}[1]{\textcolor[rgb]{0.25,0.44,0.63}{#1}}
\newcommand{\SpecialStringTok}[1]{\textcolor[rgb]{0.73,0.40,0.53}{#1}}
\newcommand{\StringTok}[1]{\textcolor[rgb]{0.25,0.44,0.63}{#1}}
\newcommand{\VariableTok}[1]{\textcolor[rgb]{0.10,0.09,0.49}{#1}}
\newcommand{\VerbatimStringTok}[1]{\textcolor[rgb]{0.25,0.44,0.63}{#1}}
\newcommand{\WarningTok}[1]{\textcolor[rgb]{0.38,0.63,0.69}{\textbf{\textit{#1}}}}
\setlength{\emergencystretch}{3em} % prevent overfull lines
\providecommand{\tightlist}{%
  \setlength{\itemsep}{0pt}\setlength{\parskip}{0pt}}
\setcounter{secnumdepth}{-\maxdimen} % remove section numbering

\author{}
\date{}

\begin{document}

\hypertarget{header-n0}{%
\section{\texorpdfstring{Include third part libs \textbf{.dll} or
\textbf{.h}}{Include third part libs .dll or .h}}\label{header-n0}}

\href{../home}{↩️ Back to index}

If you need to add some external libs to your project this is the right
section. Supposing you want add an external library called "uEye" these
are the steps to add it (remember to substitute "uEye" with your lib
name):

\begin{enumerate}
\def\labelenumi{\arabic{enumi}.}
\item
  Make the folders:

  \begin{itemize}
  \item
    \textbf{libs/uEye/libs}
  \item
    \textbf{libs/uEye/include}
  \end{itemize}
\item
  Make the file \textbf{cmake/FinduEye.cmake} containing:

\begin{Shaded}
\begin{Highlighting}[]
\CommentTok{\# Find the uEye libs}
\CommentTok{\# uEye\_FOUND {-} uEye SDK has been found on this system}
\CommentTok{\# uEye\_INCLUDE\_DIR {-} where to find the header files}
\CommentTok{\# uEye\_LIBRARY\_DIR {-} where to find the library files}

\CommentTok{\# positions where to search the libs}
\KeywordTok{SET}\NormalTok{(}
\NormalTok{  uEye\_PATH\_HINTS}
  \StringTok{"}\DecValTok{$\{PROJECT\_SOURCE\_DIR\}}\StringTok{/libs/uEye/include"}
  \StringTok{"}\DecValTok{$\{PROJECT\_SOURCE\_DIR\}}\StringTok{/libs/uEye/libs"}
\NormalTok{)}

\CommentTok{\# set uEye\_INCLUDE\_DIR}
\KeywordTok{find\_path}\NormalTok{(}
\NormalTok{  uEye\_INCLUDE\_DIR}
\NormalTok{  uEye.h}
  \OtherTok{HINTS}
  \DecValTok{$\{}\NormalTok{uEye\_PATH\_HINTS}\DecValTok{\}}
\NormalTok{)}

\CommentTok{\# set uEye\_LIBRARY\_DIR}
\KeywordTok{find\_library}\NormalTok{(}
\NormalTok{  uEye\_LIBRARY}
  \OtherTok{NAMES}
\NormalTok{  uEye\_api\_64}
  \OtherTok{HINTS}
  \DecValTok{$\{}\NormalTok{uEye\_PATH\_HINTS}\DecValTok{\}}
\NormalTok{)}

\CommentTok{\# if all listed variables are TRUE then  set uEye\_FOUND = TRUE}
\KeywordTok{include}\NormalTok{(FindPackageHandleStandardArgs)}
\KeywordTok{find\_package}\NormalTok{\_handle\_standard\_args(}
\NormalTok{  uEye}
\NormalTok{  DEFAULT\_MSG}
\NormalTok{  uEye\_LIBRARY}
\NormalTok{  uEye\_INCLUDE\_DIR}
\NormalTok{)}

\CommentTok{\# now we can use this in the others cmake files}
\KeywordTok{mark\_as\_advanced}\NormalTok{(uEye\_INCLUDE\_DIR uEye\_LIBRARY)}
\KeywordTok{set}\NormalTok{(}\DecValTok{uEye\_LIBRARIES} \DecValTok{$\{}\NormalTok{uEye\_LIBRARY}\DecValTok{\}}\NormalTok{ )}
\KeywordTok{set}\NormalTok{(}\DecValTok{uEye\_INCLUDE\_DIRS} \DecValTok{$\{}\NormalTok{uEye\_INCLUDE\_DIR}\DecValTok{\}}\NormalTok{ )}

\CommentTok{\# outputs}
\KeywordTok{message}\NormalTok{(}\StringTok{"Adding lib: uEye"}\NormalTok{)}
\KeywordTok{message}\NormalTok{(}\StringTok{"{-}{-} uEye\_FOUND: }\DecValTok{$\{uEye\_FOUND\}}\StringTok{"}\NormalTok{)}
\KeywordTok{message}\NormalTok{(}\StringTok{"{-}{-} uEye\_LIBRARIES: }\DecValTok{$\{uEye\_LIBRARIES\}}\StringTok{"}\NormalTok{)}
\KeywordTok{message}\NormalTok{(}\StringTok{"{-}{-} uEye\_INCLUDE\_DIRS: }\DecValTok{$\{uEye\_INCLUDE\_DIRS\}}\StringTok{"}\NormalTok{)}
\end{Highlighting}
\end{Shaded}
\item
  In the file \textbf{src/CMakeLists.txt} add the following lines:

\begin{Shaded}
\begin{Highlighting}[]
\CommentTok{\# Make cmake aware about the presence of Qt source files}
\CommentTok{\# set(CMAKE\_AUTOUIC ON)}
\CommentTok{\# set(CMAKE\_AUTOMOC ON)}
\CommentTok{\# set(CMAKE\_AUTORCC ON)}

\CommentTok{\# file(GLOB HEADER\_LIST CONFIGURE\_DEPENDS "$\{PROJECT\_SOURCE\_DIR\}/include/$\{PROJECT\_NAME\}/*.h")}

\CommentTok{\# Automatically check if Qt5 is installed in the system}
\CommentTok{\# Automatically find Qt5 components. REQUIRED flag: if components not found no build}
\CommentTok{\# find\_package(}
\CommentTok{\#     Qt5 COMPONENTS}
\CommentTok{\#     Core REQUIRED}
\CommentTok{\# )}

\KeywordTok{find\_package}\NormalTok{(uEye }\OtherTok{REQUIRED}\NormalTok{)}

\CommentTok{\# Create a target to build a library for the core sources.}
\CommentTok{\# Specify the sources from which the library will be built}
\CommentTok{\# Make an automatic library {-} will be static or dynamic based on user setting}
\CommentTok{\# add\_library(}
\CommentTok{\#     $\{PROJECT\_NAME\} STATIC}
\CommentTok{\#     Camera.cpp}
\CommentTok{\#     CameraFrame.cpp}
\CommentTok{\#     Cameras/IDS.cpp}
\CommentTok{\#     Cameras/Kiralux.cpp}
\CommentTok{\#     $\{HEADER\_LIST\}}
\CommentTok{\# )}

\CommentTok{\# We need this directory, and users of our library will need it too}
\CommentTok{\# target\_include\_directories($\{PROJECT\_NAME\} PUBLIC "../include")}
\KeywordTok{target\_include\_directories}\NormalTok{(}\DecValTok{$\{PROJECT\_NAME\}} \OtherTok{PRIVATE} \DecValTok{$\{}\NormalTok{uEye\_INCLUDE\_DIR}\DecValTok{\}}\NormalTok{)}

\CommentTok{\# Specify the libraries needed to build BoilerPlateQt.}
\CommentTok{\# target\_link\_libraries($\{PROJECT\_NAME\} PRIVATE Qt5::Core $\{SUBMODULES\_NAME\})}
\KeywordTok{target\_link\_libraries}\NormalTok{(}\DecValTok{$\{PROJECT\_NAME\}} \OtherTok{PRIVATE} \DecValTok{$\{uEye\_LIBRARIES\}}\NormalTok{)}
\end{Highlighting}
\end{Shaded}
\item
  Now because we include the libs using "PRIVATE" you must include the
  libs using the relative path
  \texttt{\#include\ "../../libs/uEye/include/uEye.h"}
\item
  Enjoy ;)
\end{enumerate}

\end{document}

    CMakeLists.txt is the configure root file used by cmake to set the
building process of the project.

\begin{itemize}
  \item
        \textbf{include} folder contains all the headers that must be shared
        to external project (the headers, that declare the classes and
        variables that do \textbf{not} need to be accessed through the
        library, must be placed in src or gui folders).
\end{itemize}

\begin{itemize}
  \item
        \textbf{src} folder contains all the sources for the business logic of
        the project.
\end{itemize}

\begin{itemize}
  \item
        \textbf{gui} folder contains all the sources for the definition of the
        GUI and the interaction between its widgets and the business logic.
\end{itemize}

\begin{itemize}
  \item
        \textbf{test} folder contains all the sources for testing the core and
        gui sources.
\end{itemize}

\begin{itemize}
  \item
        \textbf{apps} folder contains all the sources for executing the
        applications using the src and gui libraries.
\end{itemize}

    (*) this are Enters

\begin{lstlisting}
  git credential-wincred fill(*)
  protocol=https(*)
  host=gitlab.dei.unipd.it(*)
  username=federico(*)
  password=secret(*)(*)
\end{lstlisting}

  }
