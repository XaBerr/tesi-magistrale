\lvli{Conclusions}
This work described the realization of a PAT system for QKD experiments. Using the new QKD software structure along with the PAT repository system, the new version of PATSW was designed with the aim of providing a clear and easy-to-use system. The initial objectives were well achieved. Now the system has:
\begin{itemize}
  \item a new reliable software with a modular structure;
  \item the possibility to set many different parameters for improved exibility to different scenarios;
  \item a new board to optimize the center of the KPA101.
\end{itemize}
Both PATSW and PATHW new features have been developed, implemented and
successfully tested on the laboratory. Tests have shown that the system allows excellent performance thanks to its independence from the physical hardware. It easily adapts to 25 Hz frame rate cameras to 200 Hz frame rate cameras. The system has also shown great adaptability with actuators where movements vary from a few milliseconds to seconds. Another example of the excellent hardware independence can be seen in the fiber injection experiment where three instances of the same program were used for coarse alignment of Alice and Bob and for fine alignment of Bob.

For example, the same software will be implemented to try coupling into a single-mode fiber the light coming from a star and collected by a 400 mm telescope, as the one recently placed on the roof of our department (DEI).
Furthermore, the system offers a great potential for the implementation of further
features and facilities. Future developments could expand the system functionality easily thanks to the help of ready-to-use documentation. The software will be integrated with the algorithms and the adapters necessary for the raw pointing that will be used in links with range distances of the order 1-10 km. These will be nothing more than an extension of the core features that have been developed in this thesis, since they are not requiring additional hardware.