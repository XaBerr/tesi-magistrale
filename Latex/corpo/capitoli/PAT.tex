\lvli{Pointing Acquisition Tracking system}
\lvlii{Introduction}
\img{PAT}{Pointing, Acquisition and Tracking model system.}
To establish a communication between two non-isotropic sources (transmitter, receiver), an alignment system is required. The one used in the free-space QKD is called \textit{Pointing, Acquisition and Tracking} (PAT) system. PAT system operates on two levels: the first level is called \textit{coarse} while the second level is called \textit{fine}.
The coarse aims to align the telescopes, while the fine aims to compensate the fluctuations of the transmitted signal. The signals typically used in this communication are laser signals. Both systems can be modeled with a feedback-control shown in the \fig{PAT}.
Given a nominal position, the controller will estimate the error via a sensor. Once the estimate is completed, it will try to compensate the error by an actuator.
Now will presented a possible implementations for these systems.

\lvlii{Components}
In this section will be listed all the components used into PAT system.
\lvliii{Sensors}
\lvliv{Camera}
La camera industriale è il sensore principale per la comunicazione. A seconda del modello essa può coprire tutti gli spettri utili per la comunicazione con i laser: ultraviolet, visible, infrared. Ha tipicamente un framerate che può variare da 30Hz ai 200Hz. Il vantaggio principale della camera è che è possibile ridurre gli Hertz per aumentare il tempo di esposizione per compensare le attenuazioni tipiche delle grandi distanze. La camera ritorna una immaginine in pixel tipicamente delle dimensioni 1280x1024. Questa immagine dovrà essere poi processata dal computer per ricavare l'informazione della posizione del segnale del laser. Le camere usate nel setup del PAT sono camere IDS con tecnologia CMOS.

\lvliv{Position Sensing Detectors}
The Position Sensing Detectors (PSD) utilizes a silicon photodiode-based pincushion tetra lateral sensor to accurately measure the displacement of an incident beam relative to the calibrated center. These devices are ideal for measuring the movement of a beam, the distance traveled, or as feedback for alignment systems. Lo spettro che copre è dai 320nm ai 1100nm. Questo dispositivo necessita di un driver hardware per essere usato. A seconda della posizione del laser nel sensore, esso ritorna tre tensioni: $\Delta x, \Delta y, SUM$. Esse esprimono le intensità del laser in x, in y, e totale. È possibile usare le seguenti formule per ricavarsi la posizione del laser nel sensore di dimensioni $L_x,L_y$:
$$x = \frac{L_x \Delta x}{2 SUM}, y = \frac{L_y \Delta y}{2 SUM}.$$
Il modello usato nel PAT è il PDQ90A della ThorLabs, esso necessita di un driver chiamato KPA101.
IL PSD è tipicamente meglio della camera nelle corte distanze perché il controllore integrato nel KPA101 è basato su un circuito e quindi ha una risoluzione continua, il suo segnale non deve essere campionato ed analizzato.

\lvliv{KPA101}
Il KPA101 è un dispositivo con tre funzioni principali: fare da driver per il PSD, fare da driver per il FSM, e implementare nativamente un controllore PID. Collegandolo ad un computer da la possibilità di controllare queste tre funzioni in maniera separata. Il sistema integrato PID ha una grande limitazione, esso permette di allineare il fascio solo al centro del sistema, posizione non sempre ottimale in caso di setup complessi.


\lvliii{Actuators}
\lvliv{Fast Steering Mirror}
Il fast-steering mirror è uno specchio allogiato su una montatuta in grado di fare tip and tilt. Ha tipicamente una velocità angolare di 15 degree/s e con un range di movimento di ±2.5 degree. Il FSM usato per il PAT usa una montatura della SmarAct che sfrutta la tecnoloiga stick-slip piezo stages. The piezo actuator combine macroscopic travel with nanometer resolution and high velocities of several millimeters per second. Esso necessita di un driver esterno per essere controllato perché il suo movimento è controllato da due tensioni x e y. Il KPA101 è un driver compatibile con la tecnologia di questo attuatore.

\lvliv{Skywatcher}
Montatura per telescopio controllata da due motori elettrici

\lvliii{Controllers}
\lvliv{One Shot}
Sviluppato su pc
Controllo che necessita di una calibrazione

\lvliv{PID KPA101}
Integrato nel KPA101
Non necessita del pc per settare i parametri PID

\lvlii{Coarse}
\lvliii{Logical design}
\img{PAT-coarse}{PAT coarse model-implementation system.}
One way to point a telescope in a given region of space, it is to put a laser in that region, pointing in the direction of the telescope, and in the telescope mount a camera to see the laser.
If the camera is aligned with the telescope, then when the laser is centered in the camera, the telescope will be aligned in the desired direction.
To align two telescopes (Alice, Bob) will be needed to mount this system twice, so that Alice has a camera that points towards Bob’s laser and vice versa.

In this alignment system a camera is used as a sensor to express the system error in pixels. This error is handled by an ad-hoc developed controller that takes the name of \textit{one shot} and will be explained in the chapter \ref{chapter:2.2}. Once the controller has complete the error calculation, it will move the two-axis engine on the mount to align the telescope.

This system will maintain the aiming of both telescopes in both cases: for large movements such as the displacement of the two telescopes, and for smaller ones such as vibrations at low frequencies.

\lvliii{Physical implementation}
%  TODO: farmi dare uno schema del circuito ottico

\lvlii{Fine}
\lvliii{Logical design}
\img{PAT-fine-v1}{PAT fine model-implementation system with position sensitive device.}
\img{PAT-fine-v2}{PAT fine-v2 model-implementation system with camera.}

Once the coarse system has aligned the telescopes then the fine goes in action. To compensate for the fluctuations of the transmitted signal, a second beam is superimposed on the beam of the signal. If the wavelengths of the two beams are sufficiently close together, then both beams will be subject to the same fluctuations and thus compensate the second beam will also compensate the first.
This system is essential if you want to inject the free-space signal into a single-mode fiber. In this case the sensor can be a camera for long distances or a position sensitive device (PSD) for small distances.
The PSD can exploits another control system called \textit{PID KPA101} which will be explained in the chapter \ref{chapter:2.2}. The sensor will read the position of the laser and send it to the controller that will correct it using a fast-steering mirror (FSM). This system will be able to compensate for the atmosphere fluctuations and high frequency vibrations.

\lvliii{Physical implementation}
%  TODO: farmi dare uno schema del circuito ottico
