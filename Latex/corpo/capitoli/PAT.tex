\lvli{Pointing Acquisition Tracking system}
In physics and automation, feedback is the ability of a dynamic system to take into account the state of the system to compensate for the error by adapting the characteristics of the system itself.
For example, a pointing-acquisition-tracking (PAT) system for quantum communications requires to use an optical feedback to keep the two terminals aligned one respect to the other during the photons exchange.

This chapter starts with a brief description of the logical units comprising a free-space QKD system and then it focus on the PAT system describing its modelling and some setups in which it will be used. It concludes with a list of components and an explanation of the fine and coarse versions.

\imgs{0.4}{pictures/matera}{Picture of Matera Observatory using a pointing-acquisition-tracking system during a quantum communication experiment.}

\newpage

\lvlii{Introduction}
\img{QKD}{The requirements of QKD: PAT, Synchro, and Telecom.}
Typical free-space systems used to implement the BB84 decoy-state share the logical structure sketched in~\fig{QKD}. The three logic units necessary to implement the BB84 decoy-state protocol in a free-space channels are a Pointing-Acquisition-Tracking (\idx{PAT}) system, a Synchronization (\idx{Synchro}) system and a \idx{Telecom} interface between Alice and Bob. In this scheme \fig{QKD} Alice prepares and sends the states to Bob that measure them.

The PAT system is in charge of allowing Alice and Bob to send and receive the quantum signals over the free-space channel. It is usually comprised of a transmitting and a receiving telescope and a close-feedback pointing system to keep the two telescope aligned one with respect to the other.

The Synchro unit is in charge of the real-time timing of the protocol. It controls the lasers and tag the detections, allowing Alice and Bob to compare their raw bit strings a posteriori.

The Telecom unit is in charged of the (authenticated) classical communication between the two terminals. It allows the raw-key exchange to the other post-processing phases, the basis-reconciliation (or sifting), and error-correction and privacy-amplification~\cite{a9}.

In this thesis it will be explained how to implement the PAT system.

\lvlii{The PAT model}
\img{PAT}{Pointing, Acquisition and Tracking system's model.}
To establish a communication between two non-isotropic sources (transmitter, receiver) an alignment system is required. The one used in the free-space quantum communication is called \textit{Pointing, Acquisition and Tracking} (\idx{PAT}) system. PAT system operates on two levels: the first level is called \textit{coarse} (\idx{PATC}) while the second level is called \textit{fine} (\idx{PATF}).
The PATC aims to align the telescopes, while the PATF aims to compensate the fluctuations of the transmitted signal. The signals typically used in this communication are laser signals. Both systems can be modeled with a feedback-control shown in the~\fig{PAT}.
Given a nominal position, the controller will estimate the error via a sensor. Once the estimate is completed, it will try to compensate the error by an actuator.

The PAT system can be divide in two parts: the software (\idx{PATSW}), and the hardware (\idx{PATHW}).
PATHW can change a lot depending on what experiment you want to do, because different experiments require different components. PATSW must adapt to the PATHW but at the same time must provide an equal interface to users. This is because the goal is always the same: align the system at the desired point.

\lvlii{PAT implementations}
The PATs implemented can be multiple and depend on the specific experiment that is carried out. For example, in this thesis it will be presented in the case of a free-space QKD system with single-mode fiber coupling called PATLIC that will be detailed in chapter~\ref{cha:PATLIC experiment}. However, the same functions can also be used to try to pair single-mode fiber with light from a star, or for a generic quantum communication experiment in free-space on horizontal links.

\lvlii{Components}
In this section will be listed all the components of the PATHW which a software implementation has been made in this thesis. In addition, these components will then be used in the PATLIC experiment.
\lvliii{Sensors}
\lvliv{Camera}
\img{pictures/camera}{IDS camera.}
A camera is a common sensor used in a PAT system. Depending on the model it can cover all the useful spectra for lasers: ultraviolet, visible, infrared.
Typically, it has a frame rate that can range from 25Hz to 200Hz. The main advantage of the camera over the other sensors is its capability to reduce the frame rate to increase the exposure time to compensate for the typical attenuations of long distances.
The camera returns an image typically of order of magnitude of 1000x1000 pixels. This image will then have to be processed by the computer to calculate the location of the laser signal.
The technical details for the cameras used in the PAT setup reported in table~\ref{table:1}.

\begin{table}[h!]
      \centering
      \begin{tabular}{ |c|c| }
            \hline
            Name         & UI-3240LE           \\\hline
            Producer     & IDS                 \\\hline
            Sensor type  & CMOS                \\\hline
            Frame rate   & 30 fps              \\\hline
            Resolution   & 1280 x 1024 pixels  \\\hline
            Optical area & 6.784 mm x 5.427 mm \\\hline
            Pixel size   & 5.30 µm             \\\hline
      \end{tabular}
      \caption{Camera UI-3240LE technical details.}
      \label{table:1}
\end{table}

\lvliv{Position Sensing Detectors}
\imgs{0.2}{pictures/PSD}{ThorLabs PSD.}
The Position Sensing Detectors (\idx{PSD}) utilizes a silicon photodiode-based pincushion-tetra-lateral sensor to accurately measure the displacement of an incident beam relative to the calibrated center. These devices are ideal for measuring the movement of a beam, the distance traveled, or as feedback for alignment systems.
The spectrum covered is from 320nm to 1100nm. This device needs a hardware driver to be used. Depending on the position of the laser in the sensor, it returns three voltages: $\Delta x, \Delta y, SUM$.
From these three components it is possible to obtain the position of the laser using the following equations:
\begin{equation}
      x = \frac{L_x \cdot \Delta x}{2 \cdot SUM}, y = \frac{L_y \cdot \Delta y}{2 \cdot  SUM}
\end{equation}
where $L_x,L_y$ are the sensor sizes. The model used in the PAT is the Thorlabs PDP90A, which requires a driver called KPA101. The technical details for the PDP90A used in the PAT setup reported in table~\ref{table:2}.
The PSD is typically better than a camera at short distances, because the integrated controller in the KPA101 is based on a circuit and therefore has a continuous resolution, its signal doesn't require a post processing.

\begin{table}[h!]
      \centering
      \begin{tabular}{ |c|c| }
            \hline
            Name                 & PDP90A                 \\\hline
            Producer             & Thorlabs               \\\hline
            Wavelength range     & 320 to 1100 nm         \\\hline
            Resolutiona          & 0.75 µm                \\\hline
            Sensor size          & 9 mm x 9 mm            \\\hline
            Bandwidth            & 15 kHz                 \\\hline
            Output voltage range & pin-1 (X): -4 V to 4 V \\
                                 & pin-2 (Y): -4 V to 4 V \\
                                 & pin-3 (SUM): 0 to 4 V  \\\hline
      \end{tabular}
      \caption{Camera PDP90A technical details.}
      \label{table:2}
\end{table}

\lvliii{Drivers}
\lvliv{KPA101}
\imgs{0.25}{pictures/KPA101}{ThorLabs KPA101.}
The \idx{KPA101} is a device with three main functions: driver for the PSD, driver for the FSM, and natively implement a PID controller.
The KPA101 can be controlled by the manual interface on the top of the unit or via a USB connection to a computer running a Kinesis software package or with the PATSW. Both interfaces allow the auto aligner to be operated in either an open-loop or closed-loop mode. In both modes, the KPA101 generates a left-minus-right X difference ($\Delta X$) signal, a top-minus-bottom Y ($\Delta Y$) difference signal, and a sum (SUM) signal that are displayed on the target screen. In the closed-loop mode, a processor inside the KPA101 runs two independent feedback loops that output X DIFF and Y DIFF signals. These signals will be present at the connectors on the back of the unit for use as the inputs to the beam steering elements being used to center the beam on the detector.

\lvliii{Actuators}
\lvliv{Fast Steering Mirror}
\imgs{2.5}{pictures/FSM}{SmarAct FSM.}
The fast-steering mirror (\idx{FSM}) is a mirror housed on a mount able to do tip and tilt.
The FSM used for the PAT is mounted in a STT-25.4 SmarAct frame that takes advantage of the technology stick-slip piezo stages. The piezo actuator combine macroscopic travel with nanometer resolution and high velocities of several millimeters per second. The technical details for the STT-25.4 used in the PAT setup reported in table~\ref{table:3}.
It requires an external driver to be managed because its movement is controlled by two voltages (x,y). The KPA101 is a driver compatible with the technology of this actuator.

\begin{table}[h!]
      \centering
      \begin{tabular}{ |c|c| }
            \hline
            Name                  & STT-25.4            \\\hline
            Producer              & SmarAct             \\\hline
            Tip-tilt angles range & -2.5 to +2.5 degree \\\hline
            Angular velocity      & 15 degree/s         \\\hline
      \end{tabular}
      \caption{Fast-steering mirror STT-25.4 technical details.}
      \label{table:3}
\end{table}

\lvliv{Skywatcher}
\imgs{0.1}{pictures/skywatcher}{Skywatcher telescope mout.}
The Skywatcher used in the PAT is the AZ-GTI telescope mount. It is an altitude-azimuth mount controlled by two DC servo motor. It needs a computer to do the targeting.
It has 9 speeds of motion called rates, so to perform a movement in one of the two axes a rate-time pair should be provided. The two motors are controlled separately by two encoders that translate the rate-time pair in an angle of motion.

\lvliii{Controllers}
Ideally there are countless controllers available for the PAT system, but only two are used/developed in this thesis: One shot, and PID KPA101.
\lvliv{One shot}
\idx{One shot} is a controller developed for computers that reads the error in pixels between the input signal and the desired position, and generates a proportional correction signal for the actuator.
Since the devices are not ideal this controller will have to make a calibration for each device, in order to obtain the best voltages to be given to the various actuators.

\lvliv{PID KPA101}
The \idx{PID KPA101} controller is Proportional Integral Derivatives (\idx{PID}) and is integrated into the KPA101. A computer is required to specify PID parameters. The integrated PID system has a big limitation, it allows to align the beam only to the center of the system. This position is not always optimal in case of complex setups.

\lvliii{Additional hardware}
A problem with the integrated PID KPA101 is that aligns the system only to the center of the sensor. This becomes a problem in very complex systems when the optimal alignment position does not match the center of the sensor.
To solve this problem, a circuit was created that adds bias voltages to the signals that are exiting from the sensor and entering in the KPA101.

\lvliv{Circuit schematic}
\imgs{0.5}{circuito/schematic}{Eagle schematic view of the circuit.}
The circuit is made in Eagle. It is composed of three main components: controller, digital analog converter (DAC), amplifiers (AMP).
The controller (ESP32) is mounted on the feather-board Huzzah32 and plugs into the circuit via a socket. It talks to the DAC through the serial-peripheral-interface (SPI) protocol using the most-significant-bit (MSB) convention with two-complement-data-coding.
The output of the DAC that has a range of $\pm 2 V_{ref}$ is combined with the two input signals X, Y through two instrumentation amplifiers. The reference voltage $V_{ref} = 2.048V$ is generated by a separate integrated (ADR430ARMZ).
The amplifiers (AD8226ARMZ) has been provided with the possibility to add voltage regulators in case of amplification balance necessity.
All integrated circuits have two capacitors $0.1 \mu F, 10 \mu F$ used as power supply filters. The MISO, MOSI, SCK lines have a resistance of $1 k \Omega$ used as a pull-up.
All amplifiers have on the -IN, IN lines $100 k \Omega$ resistors as pull-down. Some jumpers have been added in order to choose whether to use the internal (KPA101) or external power supply for the integrated circuits.

\lvliv{Circuit board}
\imgTwo{0.08}{circuito/top}{Circuit top view.}{0.08}{circuito/bottom}{Circuit bottom view.}
% \img{circuito/board.pdf}{Eagle board view of the circuit.}
The track design was done using Sparkfun’s design rules circuit (DRC). Particular attention has been paid to the isolation between digital and analog circuits, and efforts have been made to maintain serial communication lines of the same length. The final circuit was tinned in the laboratory and soon will be tested.

\lvliv{Software design}
The software was created via PlatformIO. It is divided in two main parts: the first part is a communication interface between the controller and the DAC; the second part is the graphical interface that allows a user to set the bias value. Only the first part has been developed.

\lvlv{Controller-DAC interface}
\imgTwo{0.5}{circuito-input}{Simplify diagram of the sequence to write in a register.}{0.5}{circuito-output}{Simplify diagram of the sequence to read from a register.}

The communication interface between controller and DAC must implement three functions: initialization, writing from a register, and reading from a register. This interface has been implemented through the class AD5763, where the initialization is done via the constructor, and the read/write operation is done via the homonymous functions. The read and write procedures were carried out following the sequences described in the DAC documentation, their simplified schemes can be seen in~\fig{circuito-input} and~\fig{circuito-output}. Some additional structures have been implemented to simplify the use of these three functions.
For a quick view of its components, the List. \ref{code:AD5763} shows the class header.
\begin{lstlisting}[language=c++, gobble=2, label=code:AD5763]
  #ifndef AD75763_H
  #define AD75763_H
  
  #include <Arduino.h>
  #include <SPI.h>
  #include "printf.h"
  
  enum IO {
    R = 1 << 7,
    W = 0 << 7
  };
  
  enum Register {
    function   = 0 << 3,
    data       = 2 << 3,
    coarseGain = 3 << 3,
    fineGain   = 4 << 3,
    offset     = 5 << 3
  };
  
  enum DAC {
    A  = 0,
    B  = 1,
    AB = 4
  };
  
  struct PINSConfig {
    int syncNegate  = 25;
    int sclk        = 5;
    int sdin        = 18;
    int sdo         = 19;
    int clrNegate   = 16;
    int ldacNegate  = 17;
    int rstinNegate = 21;
  };
  
  class AD5763 {
    unsigned char message[3];
    PINSConfig pins;
    bool debugMode;
    SPISettings spiSettings;
    void printBin(unsigned char* message);
  
   public:
    AD5763(const PINSConfig& _pins, bool _debugMode = true);
    ~AD5763();
    void write(Register reg, int dac, const unsigned char* message);
    const unsigned char* read(Register reg, int dac);
  };
  
  #endif  // AD75763_H
\end{lstlisting}


\lvlii{PAT-Coarse system}
\img{PAT-coarse}{PATC model's implementation.}
One way to point a telescope in a given region of space is to put in that region a laser pointing in the direction of the telescope, and in the telescope mount a camera to see that laser.
If the camera is aligned with the telescope and the laser is centered in the camera, then the telescope will be aligned in the desired direction.
To align two telescopes (Alice, Bob) will be needed to mount this system twice, so that Alice has a camera that points towards Bob’s laser and vice versa.

In the PATLIC a camera is used as a sensor to express the system error in pixels. This error is handled by an ad-hoc developed controller called One shot. Once the controller has complete the error calculation, it will move the two-axis engine on the mount to align the telescope.

This system will maintain the aiming of both telescopes in both cases: for large movements such as the displacement of the two telescopes, and for smaller ones such as vibrations at low frequencies.

\lvlii{PAT-Fine system}
\img{PAT-fine-v1}{PATF model's implementation with position sensitive device.}
\img{PAT-fine-v2}{PATF model's implementation with camera.}

Once the PATC system has aligned the telescopes then the PATF goes in action. To compensate for the fluctuations of the transmitted signal, a second beam is superimposed on the beam of the signal. If the wavelengths of the two beams are sufficiently close together, then both beams will be subject to the same fluctuations and thus compensate the second beam will also compensate the first one.
This system is essential if you want to inject the free-space signal into a single-mode fiber. For example in the PATLIC experiment~\ref{cha:PATLIC experiment}, a free-space 4mm-beam should be centered in a lens to be focus on single-mode fiber, but external fluctuations that reach a centimeter range prevent the beam from having a stable injection, PATF system can greatly mitigate these fluctuations.
In this case the sensor can be a camera for long distances or a PSD for small distances.

In the PATLIC a PSD is used as a sensor and it exploits another control system PID KPA101. The sensor will read the position of the laser and send it to the controller that will correct it using a FSM.

This system will be able to compensate for the atmosphere fluctuations and high frequency vibrations.