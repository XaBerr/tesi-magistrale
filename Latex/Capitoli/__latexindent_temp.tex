\lvli{Software design}
\lvlii{The organization of the QKD software}
Tutto il codice che controllava QKD era stato realizzato in via temporanea su diversi computer senza una particolare organizzazione. Bisognava quindi implementare un sistema facile che permettesse di standardizzare, memorizzare, e semplificare l'espansione del codice.

Per memorizzare il codice è stato scelto di usare GitLab, una piattaforma che permette la memorizzazione e il versionamento del codice tramite git.

Per creare una standardizzazione è stato scelto di realizzare un Boilerplate, un codice già pronto che serve da template per la realizzazione dei vari progetti. Il Boilerplate è stato realizzato in c++ con l'integrazione di CMake e QT. CMake è un sistema che semplifica la compilazione dei sorgenti, mentre Qt è un insieme di librerie che semplificano l'interfaccia grafica per il c++. Il Boilerplate contiene inoltre due tipi di documentazione: README e Sphinx. README è un file reStructuredText (.rst) nel quale mediante un template permette di descrivere rapidamente come utilizzare la libreria. La documentazione in Sphinx è invece più complessa.

Per semplificare l'espansione del codice è stato scelto di creare un packet manager basato su CMake e di integrarlo direttamente nel Boilerplate. Con questo nuovo sistema è elencare in un file "config.json" le dipendenze di ogni progetto, ovvero le librerie necessarie alla sua compilazione. Quando il progetto viene compilato le dipendenze vengono scaricate e compilate automaticamente.

\lvlii{The development of the PAT software}
