\lvli{Software design}
\lvlii{The organization of the QKD software}
\lvliii{Memorization, standardization and expansion}
Tutto il codice che controllava QKD era stato realizzato in via temporanea su diversi computer senza una particolare organizzazione. Bisognava quindi implementare un sistema facile che permettesse di standardizzare, memorizzare, e semplificare l'espansione del codice.

Per memorizzare il codice è stato scelto di usare GitLab, una piattaforma che permette la memorizzazione e il versionamento del codice tramite git. Il codice è quindi possibile memorizzarlo in repository organizzate in gruppi.

Per creare una standardizzazione è stato scelto di realizzare un Boilerplate, un codice già pronto che serve da template per la realizzazione dei vari progetti. Il Boilerplate è stato realizzato in c++ con l'integrazione di CMake e QT. CMake è un sistema che semplifica la compilazione dei sorgenti, mentre Qt è un insieme di librerie che semplificano l'interfaccia grafica per il c++. Il Boilerplate contiene inoltre due tipi di documentazione: README e Sphinx. README è un file reStructuredText (.rst) nel quale mediante un template permette di descrivere rapidamente come utilizzare la libreria. La documentazione in Sphinx è invece più complessa. Sphinx usa una programma in python che raccoglie tutti i commenti inline nel codice e fornisce la struttura delle classi visibile tramite pagine web.

Per semplificare l'espansione del codice è stato scelto di creare un packet manager basato su CMake e di integrarlo direttamente nel Boilerplate. Con questo nuovo sistema è elencare in un file "config.json" le dipendenze di ogni progetto, ovvero le librerie necessarie alla sua compilazione. Quando il progetto viene compilato le dipendenze vengono scaricate e compilate automaticamente.

\lvliii{Boilerplate documentation}
La documentazione del Boilerplate è memorizzata in una repository separata chiamata BoilerplateQT.wiki. Essa è realizzata in markdown (.md). In questa sezione verranno elencate le sue sezioni principali.
\lvliv{Setup your environment}
\lvliii{Setup your environment}
\lvliv{Prerequisites}

These libraries need to be installed manually in your system:

\begin{itemize}
      \tightlist
      \item
            \href{https://www.qt.io/}{Qt 5.14.2}
      \item
            \href{http://www.doxygen.nl/download.html}{Doxygen 1.8.13}
      \item
            \href{https://www.anaconda.com/products/individual}{python3}
\end{itemize}

This can be installed through \textbf{pip3} using a single command line:

\begin{itemize}
      \tightlist
      \item
            \href{https://pypi.org/project/Sphinx/}{Sphinx 3.0.3}
      \item
            \href{https://sphinx-rtd-theme.readthedocs.io/en/stable/}{Sphinx read
                  the doc theme 0.4.3} (to use the read the doc theme for html
            documentation)
      \item
            \href{https://pypi.org/project/breathe/}{Breathe 4.16.0} (to use the
            xml output of doxygen)
      \item
            \href{https://pypi.org/project/sphinx-markdown-builder/}{Sphinx-markdown-builder
                  0.5.4} (to generate the markdown version for gitlab wiki)
\end{itemize}

\texttt{pip3\ install\ Sphinx\ sphinx\_rtd\_theme\ breathe\ sphinx-markdown-builder}

On \textbf{Windows} add the path of the \texttt{../Anaconda3/Scripts} folder to the system variable \textbf{PATH}.

\lvliv{Setup procedure}

\begin{itemize}
      \tightlist
      \item
            Create a new project on gitlab \{YourProject\} (the name of the
            project must be \href{https://en.wikipedia.org/wiki/Camel_case}{upper
                  camel case}).
      \item
            Download \{YourProject\} on the computer.
      \item
            Download this repository

            \texttt{git\ clone\ https://gitlab.dei.unipd.it/Calderaro/BoilerplateQT.git}.
      \item
            Copy everything except the folder \textbf{.git} from this repository
            to \{YourProject\} folder.
      \item
            (optional for QtCreator) Configure the project as described in the
            Installation.
      \item
            Replace every string \textbf{BoilerplateQt} with \{YourProject\} in
            every file (Recommended to use QtCreator advanced replace).
      \item
            Replace every string \textbf{boilerplateQt} with \{yourProject\} in
            every file (Recommended to use QtCreator advanced replace).
      \item
            Replace every string \textbf{boilerplateqt} with \{yourproject\} in
            every file (Recommended to use QtCreator advanced replace).
      \item
            Replace every string \textbf{BOILERPLATEQT} with \{YOURPROJECT\} in
            every file (Recommended to use QtCreator advanced replace).
      \item
            Rename every file that contains \textbf{boilerplateqt} with
            \{yourproject\}:
            \begin{itemize}
                  \tightlist
                  \item
                        `/gui/boilerplateqtwidget.cpp` to `/gui/{yourproject}widget.cpp`.

                  \item
                        `/gui/boilerplateqtwidget.ui` to `/gui/{yourproject}widget.ui`.

                  \item
                        `/src/boilerplateqt.cpp` to `/src/{yourproject}.cpp`.

                  \item
                        `/tests/boilerplateqttest.cpp` to `/tests/{yourproject}test.cpp`.

                  \item
                        `/include/BoilerplateQt/boilerplateqt.h` to `/include/BoilerplateQt/{yourproject}.h`.
            \end{itemize}
\end{itemize}


\begin{itemize}
      \tightlist
      \item
            Rename the folder \texttt{/include/BoilerplateQt} with
            \texttt{/include/\{YourProject\}}.
      \item
            Open \texttt{config.json} and substitute with your data:
      \item \textbf{name}: name of the project {YourProject}.

      \item \textbf{version}: last tag in {YourProject} master history.

      \item \textbf{description}: short description of {YourProject}.

      \item \textbf{GUI}: whether {YourProject} have a GUI. \textbf{If you do not have a gui remove} `/gui` \textbf{and} `/include/{YourProject}/gui` \textbf{folders}.

      \item \textbf{modules}: external libraries to be fetched from an online repositoty.
      \item
            (Optional for QtCreator):
            \begin{itemize}
                  \tightlist
                  \item Close the project on QtCreator.

                  \item Remove the file `CMakeLists.txt.user` and build directory (if any).

                  \item Configure the project as described in the Installation.
            \end{itemize}
\end{itemize}


If you correctly executed the above steps the project should build with
no problems. From this point on you can start:

\begin{itemize}
      \tightlist
      \item
            adding new source files and/or change the existing ones.
      \item
            adding and removing libraries dependeces.
\end{itemize}

\lvliv{How to add source files .cpp or .h}
\lvlv{Adding new classes to the src folder}

In the \texttt{src/CMakeLists.txt} add the source and header (only the
headers not exported in the library) files in the \textbf{add\_library}
function. If you want the class to be exported in the library, so that
an external program can use it, put the header in the
\texttt{include/ProjName} folder.


\lvlv{Adding new classes to gui folder (should not be necessary
  but why not)}

In the \texttt{gui/CMakeLists.txt} add the source and header (only the
headers not exported in the library) files in the \textbf{add\_library}
function. If you want the class to be exported in the library, so that
an external program can use it, put the header in the
\texttt{include/ProjName/gui} folder.


\lvlv{Adding new executables to the apps folder}

In the \texttt{apps/CMakeLists.txt} add the \textbf{add\_executable}
function to create a new target executable:

\begin{lstlisting}[language=c++]
  add_executable(
  "${PROJECT_NAME}NewApp"
  newapp.cpp
  )
\end{lstlisting}

Also add the libraries that should be linked to the ``NewApp'' with
target\emph{link}libraries:

\begin{lstlisting}[language=c++]
  target_link_libraries(
    "${PROJECT_NAME}NewApp"
    PRIVATE Qt5::Gui Qt5::Widgets "${PROJECT_NAME}Gui" ${PROJECT_NAME} ${SUBMODULES_NAME}
  )
\end{lstlisting}

\lvliv{Include other projects inside https://gitlab.dei.unipd.it/}
\lvliv{Include third part libs .dll or .h}
\lvliv{Structure description of the project}
\lvliv{Setup git credential in windows}

\lvlii{The development of the PAT software}
