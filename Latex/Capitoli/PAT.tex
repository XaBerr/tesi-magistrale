\lvli{Pointing Acquisition Tracking system}
\lvlii{Introduction}
Per stabilire una comunicazione tra un trasmettitore e un ricevitore non isotropici serve un sistema di allineamento. Il sistema di allineamento usato nella free-space QKD viene chiamato Pointing, Acquisition and Tracking (PAT) system. Questo sistema agisce su due livelli precisione: il primo livello viene chiamato coarse mentre il secondo livello fine. L'allineamento coarse ha l'obiettivo di allineare i telescopi mentre quello fine di compensare le fluttuazioni del segnale trasmesso tramite un laser. Entrambi i sistemi sono modellizabili con un controllo in retroazione mostrato in \fig{PAT} che cerca di allineare dei fasci laser. Data una posizione nominale di allinemanto il controllore tramite un sensore stimerà l'errore che cerca di correggere tramite un'attuatore.
\img{PAT}{Pointing, Acquisition and Tracking model system.}



\lvlii{Coarse}
\img{PAT-coarse}{PAT coarse model system.}
Per fare in modo che un telescopio punti un una determinata regione dello spazio, in quella regione si mette un laser divergente che punta nella direzione del telescopio, e nel telescopio si monta una camera. Se la camera è allineata rispetto al telescopio, allora quando si cerca di centrare il laser nella camera, si allinerà anche il telescopio nella direzione voluta. Per allineare quindi due telescopi (Alice, Bob) bisognerà montare questo sistema doppio in modo che Alice abbia una camera che punti verso il laser di Bob e vice versa.

è possibile implementare diversi controllori
la camera funziona in px

% Il sistema di allineamento viene gestito separatamente nei due telescopi. L'allineamento viene gestito da un sistema in retroazione composto da tre elementi: sensore, controllore, e attuatore. Il sensore che in questo caso è la camera, leggerà la posizione del laser dell'altro telescopio e la manderà ad un controllore che correggerà la posizione del telescopio usando due motori ad alta precisione.

\lvlii{Fine}
Una volta che il sistema coarse ha allineato i telescopi il sistema fine entra in azione. Il compito del sistema fine è quello di compensare le fluttuazioni del laser dovute all'atmosfera e alle vibrazioni meccaniche. Questo secondo sistema è fondamentale se si vuole accoppiare il segnale in free-space in fibra a singolo modo. Anche questo sistema si basa su un controllo in retroazione composto da tre elementi: sensore, controllore, e attuatore. Il sensore in questo caso può essere una camera per le grandi distanze oppure un PSD per le piccole. Il sensore leggerà la posizione del laser e la manderà al controllore che la correggerà usando un fast-steering mirror.