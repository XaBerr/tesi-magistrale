\lvliii{Camera}

This project contains the interface \textbf{Camera} for all cameras. The
actual implementations are: IDS.

\textbf{Requirements}

\includegraphics[scale=0.7]{img/shilds/cpp.png}
\includegraphics[scale=0.7]{img/shilds/cmake.png}
\includegraphics[scale=0.7]{img/shilds/git.png}
\includegraphics[scale=0.7]{img/shilds/doxygen.png}
\includegraphics[scale=0.7]{img/shilds/sphinx.png}
\includegraphics[scale=0.7]{img/shilds/win.png}

\lvliv{Generality}

\lvlv{Import}

Import as an external library into your project by copy-paste the
following lines in your \texttt{config.json}.

\begin{lstlisting}[language=javascript, gobble=2]
  {
    "name"     : "PATCamera",
    "path"     : "gitlab.dei.unipd.it/PAT/Camera.git",
    "tag"      : "HEAD",
    "available": "YES",
    "getGui"   : "YES"
  }
\end{lstlisting}

\lvlv{Prerequisites}

The following libraries are auto fetched from the gitlab.dei.unipd.it
host (ask the owner of this repo to become a member):

\begin{itemize}
  \tightlist
  \item
        \href{https://gitlab.dei.unipd.it/PAT/Centroid.git}{Centroid} 1.0.0
\end{itemize}

These other libraries need to be installed manually in your system:

\begin{itemize}
  \tightlist
  \item
        \href{https://www.qt.io/}{Qt} 5.14.2
\end{itemize}

The library documentation is generated through
\href{http://www.doxygen.nl/download.html}{Doxygen 1.8.13}. Additional
documentation in the \texttt{index} folder is generated through the
\href{https://www.anaconda.com/products/individual}{python3} package
\href{https://www.sphinx-doc.org/en/master/}{Sphinx} using the following
extensions (which you can install through pip3):

\begin{itemize}
  \tightlist
  \item
        \href{https://pypi.org/project/Sphinx/}{Sphinx 3.0.2}
  \item
        \href{https://sphinx-rtd-theme.readthedocs.io/en/stable/}{Sphinx read
          the doc theme} (to use the read the doc theme for html documentation)
  \item
        \href{https://pypi.org/project/breathe/}{Breathe} (to use the xml
        output of doxygen)
  \item
        \href{https://pypi.org/project/sphinx-markdown-builder/}{Sphinx-markdown-builder}
        (to generate the markdown version for gitlab wiki)
\end{itemize}

\texttt{pip3\ install\ Sphinx\ sphinx\_rtd\_theme\ breathe\ sphinx-markdown-builder}

\lvliv{Usage}

\lvlv{Frame}

The \textbf{Frame} class manages all the processing of the images
captured by the camera.

\begin{lstlisting}[language=c++, gobble=2]
  #include <PAT/Camera/Frame.h>
\end{lstlisting}

This is an example of initialization.

\begin{lstlisting}[language=c++, gobble=2]
  int width = 100;
  int height = 100;
  unsigned char imgBuffer[width * height] = {0};
  std::shared_ptr<Frame> frame = std::make_shared<Frame>(imgBuffer, width, height, QImage::Format_Grayscale8);
\end{lstlisting}

You can flip the image in the two axes usign these functions.

\begin{lstlisting}[language=c++, gobble=2]
  frame->flipX();
  frame->flipY();
\end{lstlisting}

You can calculate the centroid, the spotDiameter and the integral (spot
intensity) using this function. If it does not find the centroid it
returns -1 as invalid value.

\begin{lstlisting}[language=c++, gobble=2]
  frame->calculateCentroidAndIntegral();
\end{lstlisting}

You can also control the centroid calculation setting the parameters
\textbf{threshold} and \textbf{rangeRadius}. -
\textbf{setThreshold/getThreshold} set/get the minimum value for pixels
to be counted in the centroid average. -
\textbf{setRangeRadius/getRangeRadius} set/get the range radius in pixel
where to search the centroid. The value 0 means infinite range. -
\textbf{getRange} get the four cordinate of the range where to search
the centroid.

\begin{lstlisting}[language=c++, gobble=2]
  auto threshold = frame->getThreshold();
  threshold.value = 30;
  frame->setThreshold(threshold);
  
  auto rangeRadius = frame->getRangeRadius();
  rangeRadius.value = 200;
  frame->setRangeRadius(rangeRadius);
  
  PAT::Utils::Range<int> range = frame->getRange();
\end{lstlisting}

You can get the other parameters using these functions.

\begin{lstlisting}[language=c++, gobble=2]
  std::shared_ptr<PAT::Centroid> centroid = frame->getCentroid();
  std::shared_ptr<PAT::Utils::BoundedParameter<PAT::Utils::Circle<int>>> target = frame->getTarget();
  unsigned char* imgBuffer = frame->getImageRaw();
  QImage img = frame->getImage();
  int width = frame->getWidth();
  int height = frame->getHeight();
\end{lstlisting}

\lvlv{Camera}

The \textbf{Camera} class is a interface for all Camera. Here is the
include code.

\begin{lstlisting}[language=c++, gobble=2]
  #include <PAT/Camera/Camera.h>
\end{lstlisting}

To inherit the class you have to overwrite the four methods:
\textbf{start}, \textbf{stop}, \textbf{setFrameRate},
\textbf{setExposureTime}.

\begin{lstlisting}[language=c++, gobble=2]
  class YourClass : public Camera {
    public:
      void start() override;
      void stop() override;
      void setFrameRate(const PAT::Utils::BoundedParameter<double>& _frameRate) override;
      void setExposureTime(const PAT::Utils::BoundedParameter<double>& _exposureTimeValue) override;
    };
\end{lstlisting}

\lvlv{IDS}

The \textbf{IDS} class is a Camera implementation. Here is the include
code.

\begin{lstlisting}[language=c++, gobble=2]
  #include <PAT/Camera/IDS.h>
  using namespace PAT::Camera;
\end{lstlisting}

The \textbf{IDS} is typically used as polymorphism with \textbf{Camera}.
In order to define a variable you can choose to pass the \textbf{IDS id}
or to let it automatically search it. If the search fails the program
throw an exception "not found". To search the id you can use
\textbf{find} and \textbf{findAll} functions.

\begin{lstlisting}[language=c++, gobble=2]
  std::shared_ptr<PAT::Camera::Camera> camera = std::make_shared<PAT::Camera::IDS>();
  std::shared_ptr<PAT::Camera::Camera> camera = std::make_shared<PAT::Camera::IDS>(IDS::find());
  std::shared_ptr<PAT::Camera::Camera> camera = std::make_shared<PAT::Camera::IDS>(IDS::findAll()->uci[0].dwCameraID);
\end{lstlisting}

You can enable/disable or check if enabled the IDS's acquisition using
these functions.

\begin{lstlisting}[language=c++, gobble=2]
  if(!camera->isActive())
    camera->start();
 else
    camera->stop();
\end{lstlisting}

You can set the frame rate of the acquisition using these functions. At
every frame the Camera calculates the centroid and then it emit a signal
\textbf{newFrameSignal}.

\begin{lstlisting}[language=c++, gobble=2]
  auto framerate  = camera->getFrameRate();
  framerate.value = 30;
  camera->setFrameRate(framerate);
\end{lstlisting}

You can set the exposure time of the acquisition using these functions.
The exposure time limit depends on the frame rate (1 / framerate).

\begin{lstlisting}[language=c++, gobble=2]
  auto exposureTime  = camera->getExposureTime();
  exposureTime.value = 1 / 30;
  camera->setExposureTime(exposureTime);
\end{lstlisting}

You can enable/disable or check if enabled the centroid calculation
using these functions.

\begin{lstlisting}[language=c++, gobble=2]
  if(!camera->getCentroidCalculation())
    camera->setCentroidCalculation(true);
 else
    camera->setCentroidCalculation(false);
\end{lstlisting}

You can alse obtain the frame object that it automatically update at
every frame.

\begin{lstlisting}[language=c++, gobble=2]
  std::shared_ptr<Frame> frame = camera->getFrame();
\end{lstlisting}

\lvlv{CameraWiget}

The \textbf{CameraWiget} class is a widget fot the Camera's
implementations. Here I present the IDS example. The include code is the
following.

\begin{lstlisting}[language=c++, gobble=2]
  #include <QApplication>
  #include <PAT/Camera/gui/CameraWidget.h>
  #include <PAT/Camera/IDS.h>
  using namespace PAT::Camera;
\end{lstlisting}

This is a simple \textbf{main} example.

\begin{lstlisting}[language=c++, gobble=2]
  int main(int argc, char *argv[]) {
    QApplication a(argc, argv);
    CameraWidget cameraWidget;
    std::shared_ptr<Camera> camera = std::make_shared<IDS>();
    cameraWidget.setCamera(camera);
    cameraWidget.show();
    camera->start();
   return a.exec();
  }
\end{lstlisting}

If you want instead integrate the widget into another GUI (layout) you
can use this code.

\begin{lstlisting}[language=c++, gobble=2]
  QWidget *window                = new QWidget();
  QVBoxLayout *mainLayout        = new QVBoxLayout();
  CameraWidget *cameraWidget     = new CameraWidget();
  std::shared_ptr<Camera> camera = std::make_shared<IDS>();
  
  cameraWidget->setCamera(camera);
  mainLayout->addWidget((QWidget *)cameraWidget);
  window->setLayout(mainLayout);
  window->show();
\end{lstlisting}

\lvliv{UML}

\img{Camera}{Camera UML.}
