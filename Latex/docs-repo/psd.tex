\lvliii{PSD}

This project contains the interface \textbf{PSD} for all PSD. The actual
implementations are: PDP90A.

\textbf{Requirements}

\includegraphics[scale=0.7]{img/shilds/cpp.png}
\includegraphics[scale=0.7]{img/shilds/cmake.png}
\includegraphics[scale=0.7]{img/shilds/git.png}
\includegraphics[scale=0.7]{img/shilds/doxygen.png}
\includegraphics[scale=0.7]{img/shilds/sphinx.png}
\includegraphics[scale=0.7]{img/shilds/win.png}

\lvliv{Generality}

\lvlv{Import}

Import as an external library into your project by copy-paste the
following lines in your \texttt{config.json}.

\begin{lstlisting}[language=javascript, gobble=2]
  {
    "name"     : "PATPSD",
    "path"     : "gitlab.dei.unipd.it/PAT/PSD.git",
    "tag"      : "HEAD",
    "available": "YES",
    "getGui"   : "YES"
  }
\end{lstlisting}

\lvlv{Prerequisites}

The Prerequisites are the same descibed in the section "Setup your enviroment Prerequisites" \sect{Setup your environment} with the addition of the following libraries that are auto fetched from the gitlab.dei.unipd.it:

\begin{itemize}
  \tightlist
  \item
        \href{https://gitlab.dei.unipd.it/PAT/KPA101.git}{KPA101} 1.0.0
  \item
        \href{https://gitlab.dei.unipd.it/PAT/Centroid.git}{Centroid} 1.0.0
\end{itemize}


\lvliv{UML}
In figure \fig{PSD} is illustrated the UML of the repository that shows the relationships between the classes that will be describe below.
\img{PSD}{PSD UML.}

\lvliv{Usage}

\lvlv{PSD}

The \textbf{PSD} class is a interface for all PSD. Here is the include
code.

\begin{lstlisting}[language=c++, gobble=2]
  #include <PAT/PSD/PSD.h>
\end{lstlisting}

To inherit the class you have to overwrite the three methods:
\textbf{start}, \textbf{stop}, \textbf{getIntensity}.

\begin{lstlisting}[language=c++, gobble=2]
  class YourClass : public PSD {
    public:
      void start() override;
      void stop() override;
      double getIntensity() const override;
    };    
\end{lstlisting}

\lvlv{PDP90A}

The \textbf{PDP90A} class is a PSD implementation that works with the
\textbf{KPA101}. Here is the include code.

\begin{lstlisting}[language=c++, gobble=2]
  #include <PAT/KPA101.h>
  #include <PAT/PSD/PDP90A.h>
  using namespace PAT::PSD;  
\end{lstlisting}

When you istantiate the \textbf{PDP90A} you must pass a \textbf{KPA101}
istance. The \textbf{PDP90A} is typically used as polymorphism with
\textbf{PSD}.

\begin{lstlisting}[language=c++, gobble=2]
  std::shared_ptr<PAT::KPA101> kpa101   = std::make_shared<PAT::KPA101>();
  std::shared_ptr<PAT::PSD::PSD> psd = std::make_shared<PAT::PSD::PDP90A>(kpa101);
\end{lstlisting}

You can enable/disable or check if enabled the PDP90A's acquisition
using these functions.

\begin{lstlisting}[language=c++, gobble=2]
  if(!psd->isActive())
    psd->start();
 else
    psd->stop();
\end{lstlisting}

You can set the frame rate of the acquisition using these functions. At
every frame the PSD calculates the centroid and then it emit a signal
\textbf{newFrameSignal}.

\begin{lstlisting}[language=c++, gobble=2]
  auto framerate  = psd->getFrameRate();
  framerate.value = 30;
  psd->setFrameRate(framerate);
\end{lstlisting}

You can get the other parameters using these functions.

\begin{lstlisting}[language=c++, gobble=2]
  PAT::Utils::Point<double> sensorSize = psd->getSensorSize();
  std::shared_ptr<PAT::Utils::BoundedParameter<PAT::Utils::Circle<int>>> target = psd->getTarget();
  std::shared_ptr<PAT::Centroid> centroid = psd->getCentroid();
  PAT::Utils::Point<double> sensorSize = psd->getSensorSize();
  double intensity = psd->getIntensity();
\end{lstlisting}

\lvlv{PSDWiget}

The \textbf{PSDWiget} class is a widget fot the PSD's implementations.
Here I present the PDP90A example. The include code is the following.

\begin{lstlisting}[language=c++, gobble=2]
  #include <QApplication>
  #include <PAT/PSD/gui/PSDWidget.h>
  #include <PAT/PSD/PDP90A.h>
  using namespace PAT::PSD;
\end{lstlisting}

This is a simple \textbf{main} example.

\begin{lstlisting}[language=c++, gobble=2]
  int main(int argc, char *argv[]) {
    QApplication a(argc, argv);
    PSDWidget cameraWidget;
    std::shared_ptr<PAT::KPA101> kpa101 = std::make_shared<PAT::KPA101>();
    std::shared_ptr<PSD> camera         = std::make_shared<PDP90A>(kpa101);
    cameraWidget.setPSD(camera);
    cameraWidget.show();
    camera->start();
    return a.exec();
 }
\end{lstlisting}

If you want instead integrate the widget into another GUI (layout) you
can use this code.

\begin{lstlisting}[language=c++, gobble=2]
  QWidget *window                     = new QWidget();
  QVBoxLayout *mainLayout             = new QVBoxLayout();
  PSDWidget *psdWidget                = new PSDWidget();
  std::shared_ptr<PAT::KPA101> kpa101 = std::make_shared<PAT::KPA101>();
  std::shared_ptr<PSD> camera         = std::make_shared<PDP90A>(kpa101);
  
  psdWidget->setPSD(psd);
  mainLayout->addWidget((QWidget *)psdWidget);
  window->setLayout(mainLayout);
  window->show();
\end{lstlisting}

