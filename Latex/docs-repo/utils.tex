\lvliii{Utils}

This project contains a set of utilities for the PAT's repositories.
This REPO contains:

\begin{itemize}
  \tightlist
  \item
        Interval
  \item
        Point
  \item
        Range
  \item
        BoundedParameter
\end{itemize}

\textbf{Requirements}

\includegraphics[scale=0.7]{img/shilds/cpp.png}
\includegraphics[scale=0.7]{img/shilds/cmake.png}
\includegraphics[scale=0.7]{img/shilds/git.png}
\includegraphics[scale=0.7]{img/shilds/doxygen.png}
\includegraphics[scale=0.7]{img/shilds/sphinx.png}
\includegraphics[scale=0.7]{img/shilds/win.png}
\includegraphics[scale=0.7]{img/shilds/mac.png}
\includegraphics[scale=0.7]{img/shilds/linux.png}

\lvliv{Generality}

\lvlv{Import}

Import as an external library into your project by copy-paste the
following lines in your \texttt{config.json}.

\begin{lstlisting}[language=javascript, gobble=2]
  {
    "name"     : "PATUtils",
    "path"     : "gitlab.dei.unipd.it/PAT/Utils.git",
    "tag"      : "HEAD",
    "available": "YES",
    "getGui"   : "NO"
  }
\end{lstlisting}


\lvlv{Prerequisites}

These libraries need to be installed manually in your system:

\begin{itemize}
  \tightlist
  \item
        \href{https://www.qt.io/}{Qt} 5.14.2
\end{itemize}

The library documentation is generated through
\href{http://www.doxygen.nl/download.html}{Doxygen 1.8.13}. Additional
documentation in the \texttt{index} folder is generated through the
\href{https://www.anaconda.com/products/individual}{python3} package
\href{https://www.sphinx-doc.org/en/master/}{Sphinx} using the following
extensions (which you can install through pip3):

\begin{itemize}
  \tightlist
  \item
        \href{https://pypi.org/project/Sphinx/}{Sphinx 3.0.2}
  \item
        \href{https://sphinx-rtd-theme.readthedocs.io/en/stable/}{Sphinx read
          the doc theme} (to use the read the doc theme for html documentation)
  \item
        \href{https://pypi.org/project/breathe/}{Breathe} (to use the xml
        output of doxygen)
  \item
        \href{https://pypi.org/project/sphinx-markdown-builder/}{Sphinx-markdown-builder}
        (to generate the markdown version for gitlab wiki)
\end{itemize}

\texttt{pip3\ install\ Sphinx\ sphinx\_rtd\_theme\ breathe\ sphinx-markdown-builder}

\lvliv{Usage}

\lvlv{Interval}

Interval give the possibility to generate a base-type structure that it
has a \textbf{min} and \textbf{max} value.

First you need to include the header file and use the namespace.

\begin{lstlisting}[language=c++, gobble=2]
  #include <PAT/Utils/Interval.h>
  using namespace PAT::Utils;
\end{lstlisting}

Then you can define the variable ready to be used.

\begin{lstlisting}[language=c++, gobble=2]
  Interval<double> interval{0, 100};
  interval.min = 0;
  interval.max = 100;
\end{lstlisting}

\lvlv{Point}

Point give the possibility to generate a 2d-point that it has a
\textbf{x} and \textbf{y} value.

First you need to include the header file and use the namespace.

\begin{lstlisting}[language=c++, gobble=2]
  #include <PAT/Utils/Point.h>
  using namespace PAT::Utils;
\end{lstlisting}

Then you can define the variable ready to be used.

\begin{lstlisting}[language=c++, gobble=2]
  Point<double> point{0, 100};
  point.x = 0;
  point.y = 100;  
\end{lstlisting}

It has some operations already defined suc as: sum, division, scalar
product.

\begin{lstlisting}[language=c++, gobble=2]
  Point<double> point2{0, 100};
  point += point2;
  point -= point2;
  point /= 3;
  point *= 3;  
\end{lstlisting}

\lvlv{Rectangle}

Rectangle give the possibility to generate a rectangular range defined
by two points(\textbf{Point}) that they are \textbf{start} and
\textbf{end}.

First you need to include the header file and use the namespace.

\begin{lstlisting}[language=c++, gobble=2]
  #include <PAT/Utils/Rectangle.h>
  using namespace PAT::Utils;
\end{lstlisting}

Then you can define the variable ready to be used.

\begin{lstlisting}[language=c++, gobble=2]
  Rectangle<double> rectangle{{0, 0}, {100, 100}};
  rectangle.start.x = 0;
  rectangle.start.y = 0;
  rectangle.end.x = 100;
  rectangle.end.y = 100;
  Point<double> point{50, 50};
  if(rectangle.isInside(point))
   rectangle.end = point;
\end{lstlisting}

\lvlv{Circle}

Circle give the possibility to generate a circular range defined by a
central point(Point) and a radius.

First you need to include the header file and use the namespace.

\begin{lstlisting}[language=c++, gobble=2]
  #include <PAT/Utils/Circle.h>
  using namespace PAT::Utils;
\end{lstlisting}

Then you can define the variable ready to be used.

\begin{lstlisting}[language=c++, gobble=2]
  Circle<double> circle{{0, 0}, 100};
  circle.center.x = 0;
  circle.center.y = 0;
  circle.radius = 100;
  Point<double> point{50, 50};
  if(circle.isInside(point))
    circle.radius = point.x;
  Rectangle<double> boundaries = circle.getBoundaries();
\end{lstlisting}

\lvlv{BoundedParameter}

BoundedParameter give the possibility to define a parameter that it has:
\textbf{min} value, \textbf{max} value, \textbf{increment} value, and
\textbf{value} value.

First you need to include the header file and use the namespace.

\begin{lstlisting}[language=c++, gobble=2]
  #include <PAT/Utils/BoundedParameter.h>
  using namespace PAT::Utils;
\end{lstlisting}

Then you can define the variable ready to be used.

\begin{lstlisting}[language=c++, gobble=2]
  BoundedParameter<double> parameter{0, 100, 1, 50};
  parameter.min = 0;
  parameter.max = 100;
  parameter.increment = 1;
  parameter.value = 50;
\end{lstlisting}

\textbf{Warning}

The order of these parameters is imporant because is the same of the IDS
camera (min, max, increment).

\lvliv{UML}

\img{utils}{Utils UML.}
