\lvliii{KPA101}

This project create and interface between sensors, controllers,
actuators and all the functionalities of the KPA101.

\textbf{Requirements}

\includegraphics[scale=0.7]{img/shilds/cpp.png}
\includegraphics[scale=0.7]{img/shilds/cmake.png}
\includegraphics[scale=0.7]{img/shilds/git.png}
\includegraphics[scale=0.7]{img/shilds/doxygen.png}
\includegraphics[scale=0.7]{img/shilds/sphinx.png}
\includegraphics[scale=0.7]{img/shilds/win.png}

\lvlv{Generality}

\lvliv{Import}

Import as an external library into your project by copy-paste the
following lines in your \texttt{config.json}.

\begin{lstlisting}[language=javascript, gobble=2]
  {
    "name"     : "PATKPA101",
    "path"     : "gitlab.dei.unipd.it/PAT/KPA101.git",
    "tag"      : "HEAD",
    "available": "YES",
    "getGui"   : "NO"
  }
\end{lstlisting}

\lvliv{Prerequisites}

The following libraries are auto fetched from the gitlab.dei.unipd.it
host (ask the owner of this repo to become a member):

\begin{itemize}
  \tightlist
  \item
        \href{https://gitlab.dei.unipd.it/PAT/Utils.git}{Utils} 1.0.0
\end{itemize}

These other libraries need to be installed manually in your system:

\begin{itemize}
  \tightlist
  \item
        \href{https://www.qt.io/}{Qt} 5.14.2
\end{itemize}

The library documentation is generated through
\href{http://www.doxygen.nl/download.html}{Doxygen 1.8.13}. Additional
documentation in the \texttt{index} folder is generated through the
\href{https://www.anaconda.com/products/individual}{python3} package
\href{https://www.sphinx-doc.org/en/master/}{Sphinx} using the following
extensions (which you can install through pip3):

\begin{itemize}
  \tightlist
  \item
        \href{https://pypi.org/project/Sphinx/}{Sphinx 3.0.2}
  \item
        \href{https://sphinx-rtd-theme.readthedocs.io/en/stable/}{Sphinx read
          the doc theme} (to use the read the doc theme for html documentation)
  \item
        \href{https://pypi.org/project/breathe/}{Breathe} (to use the xml
        output of doxygen)
  \item
        \href{https://pypi.org/project/sphinx-markdown-builder/}{Sphinx-markdown-builder}
        (to generate the markdown version for gitlab wiki)
\end{itemize}

\texttt{pip3\ install\ Sphinx\ sphinx\_rtd\_theme\ breathe\ sphinx-markdown-builder}

\lvlv{Usage}

\lvliv{InputKPA}

InputKPA is a utility structure that contains the three voltagies that
KPA101 takes in input: \textbf{x}, \textbf{y}, \textbf{sum}.

First you need to include the header file and use the namespace.

\begin{lstlisting}[language=c++, gobble=2]
  #include <PAT/InputKPA.h>
  using namespace PAT;  
\end{lstlisting}

Then you can define the variable ready to be used.

\begin{lstlisting}[language=c++, gobble=2]
  InputKPA parameters;
  parameters.x = 1;
  parameters.y = 2;
  parameters.sum = 3;
\end{lstlisting}

\lvliv{KPA101}

KPA101 is a wrapper for the Kinesis libs that enhance their
functionalities.

First you need to include the header file and use the namespace.

\begin{lstlisting}[language=c++, gobble=2]
  #include <PAT/KPA101.h>
  using namespace PAT;
\end{lstlisting}

In order to define a variable you can choose to pass the \textbf{kpa id}
or to let it automatically search it. If the search fails the program
throw an exception "not found". To search the id you can use
\textbf{find} and \textbf{findAll} functions.

\begin{lstlisting}[language=c++, gobble=2]
  KPA101 kpa101;          // auto
  KPA101 kpa101("kpaid"); // with id
  KPA101 kpa101(KPA101::find());       // with find
  KPA101 kpa101(KPA101::findAll()[0]); // with findAll
\end{lstlisting}

\textbf{Sensor}

Here are listed the functions useful for sensors.

If you need the get the current input state you can use this function.

\begin{lstlisting}[language=c++, gobble=2]
  InputKPA getInputVoltage() const;
\end{lstlisting}

\textbf{Actuator}

Here are listed the functions useful for actuators.

You can set the output state of the KPA using one of these three
functions. Output voltages are between 10 and 0.

\begin{itemize}
  \tightlist
  \item
        The function \textbf{setOutputVoltage} set the output voltages at the
        selected values.
  \item
        The function \textbf{setOutputImpulse} send a single impulse with
        length \textbf{\_time\_ms}.
  \item
        The function \textbf{setOutputCombinedImpulse} send two impulses at
        the same time with with their respective lengths. I one finish before
        the other its voltage go to zero.
\end{itemize}

\begin{lstlisting}[language=c++, gobble=2]
  double voltageX = 1;
  double voltageY = 1;
  kpa101.setOutputVoltage(voltageX, voltageY);

  double voltagePeakX = 1;
  double voltagePeakY = 1;
  double time_ms      = 1;
  kpa101.setOutputImpulse(voltagePeakX, voltagePeakY, time_ms);

  double voltagePeakX = 1;
  double timeX_ms     = 1;
  double voltagePeakY = 1;
  double timeY_ms     = 1;
  kpa101.setOutputCombinedImpulse(voltagePeakX, timeX_ms, voltagePeakY, timeY_ms);
\end{lstlisting}

\textbf{Controller}

Here are listed the functions useful to use the PID controller
integrated in the KPA.

You can set the PID parameters using these functions. PID values are
between 1 and 0.

\begin{lstlisting}[language=c++, gobble=2]
  auto parameters = kpa101.getPIDParameters();
  parameters.value.proportionalGain = 1;
  parameters.value.integralGain = 0.001;
  parameters.value.differentialGain = 0.01;
  kpa101.setPIDParameters(parameters);
\end{lstlisting}

You can enable/disable or check if enabled the PID controller using
these functions.

\begin{lstlisting}[language=c++, gobble=2]
  if(!kpa101.isPIDActive())
    kpa101.enablePID(true);
  else
    kpa101.enablePID(false);
\end{lstlisting}

You can enable/disable the auto open closed loop. If the intensity of
the light is outside the interval \textbf{autoOpenInterval} then the PID
controller is stopped. The intensity value is bounded between 10 and 0.

\begin{lstlisting}[language=c++, gobble=2]
  auto interval = kpa101.getAutoOpenInterval();
  interval.value.min = 0.1;
  interval.value.max = 1.3;
  kpa101.setAutoOpenInterval(interval);
  if(!kpa101.getAutoOpenCloseLoop())
    kpa101.setAutoOpenCloseLoop(true);
\end{lstlisting}

The output of the PID controller is multiplied by a feedback constant
that range between -10 to +10 volts. Using this function you can set
it.

\begin{lstlisting}[language=c++, gobble=2]
  auto feedback = kpa101.getFeedBackGain();
  gain.value.x = 10;
  gain.value.y = 10;
  kpa101.setFeedBackGain(feedback);
\end{lstlisting}

\lvlv{UML}

\img{KPA101}{KPA101 UML.}