\lvliii{Controller}

This project contains different controllers for the PAT system. The
controllers implemented at moment are: Manual, Calibrated, PIDKPA101.

\textbf{Requirements}

\includegraphics[scale=0.7]{img/shilds/cpp.png}
\includegraphics[scale=0.7]{img/shilds/cmake.png}
\includegraphics[scale=0.7]{img/shilds/git.png}
\includegraphics[scale=0.7]{img/shilds/doxygen.png}
\includegraphics[scale=0.7]{img/shilds/sphinx.png}
\includegraphics[scale=0.7]{img/shilds/win.png}

\lvliv{Generality}

\lvlv{Import}

Import as an external library into your project by copy-paste the
following lines in your \texttt{config.json}.

\begin{lstlisting}[language=javascript, gobble=2]
    {
      "name"     : "PATController",
      "path"     : "gitlab.dei.unipd.it/PAT/Controller.git",
      "tag"      : "HEAD",
      "available": "YES",
      "getGui"   : "YES"
    }
\end{lstlisting}

\lvlv{Prerequisites}

The Prerequisites are the same descibed in the section "Setup your enviroment Prerequisites" \sect{Setup your environment} with the addition of the following libraries that are auto fetched from the gitlab.dei.unipd.it:

\begin{itemize}
  \tightlist
  \item
        \href{https://gitlab.dei.unipd.it/PAT/Mirror}{Mirror} 1.0.0
  \item
        \href{https://gitlab.dei.unipd.it/PAT/PSD}{PSD} 1.0.0
  \item
        \href{https://gitlab.dei.unipd.it/PAT/Camera}{Camera} 1.0.0
\end{itemize}

\lvliv{UML}
In figure \fig{Controller} is illustrated the UML of the repository that shows the relationships between the classes that will be describe below.
\img{Controller}{Controller UML.}

\lvliv{Usage}

\lvlv{PIDKPA101}

This class offers a set of functions to comunicate with the PID
controller integrated in the KPA101.

First you need to include the header file and use the namespace.

\begin{lstlisting}[language=c++, gobble=2]
  #include <PAT/Controller/PIDKPA101.h>
  using namespace PAT::Controller;
\end{lstlisting}

When you istantiate the \textbf{PIDKPA101} you must pass a
\textbf{KPA101} istance.

\begin{lstlisting}[language=c++, gobble=2]
  std::shared_ptr<PAT::KPA101> kpa101   = std::make_shared<PAT::KPA101>();
  std::shared_ptr<PAT::Controller::PIDKPA101> pidkpa101 = std::make_shared<PAT::Controller::PIDKPA101>(kpa101);
\end{lstlisting}


You can enable/disable or check if enabled the PIDKPA101 controller
using these functions.

\begin{lstlisting}[language=c++, gobble=2]
  if(!pidkpa101->isPIDActive())
    pidkpa101->enablePID(true);
 else
    pidkpa101->enablePID(false); 
\end{lstlisting}


You can set the PID parameters using these functions. PID values are
between 1 and 0.

\begin{lstlisting}[language=c++, gobble=2]
  auto parameters = pidkpa101->getPIDParameters();
  parameters.value.proportionalGain = 1;
  parameters.value.integralGain = 0.001;
  parameters.value.differentialGain = 0.01;
  pidkpa101->setPIDParameters(parameters);
\end{lstlisting}


You can enable/disable the auto open closed loop. If the intensity of
the light is outside the interval \textbf{autoOpenInterval} then the PID
controller is stopped. The intensity value is bounded between 10 and 0.

\begin{lstlisting}[language=c++, gobble=2]
  auto interval = pidkpa101->getAutoOpenInterval();
  interval.value.min = 0.1;
  interval.value.max = 1.3;
  pidkpa101->setAutoOpenInterval(interval);
  if(!pidkpa101->getAutoOpenCloseLoop())
   pidkpa101->setAutoOpenCloseLoop(true);
\end{lstlisting}

The output of the PID controller is multiplied by a feedback constant
that range between -10 to +10 volts. Using this function you can set
it.

\begin{lstlisting}[language=c++, gobble=2]
  auto feedback = pidkpa101->getFeedBackGain();
  gain.value.x = 10;
  gain.value.y = 10;
  pidkpa101->setFeedBackGain(feedback);
\end{lstlisting}


\lvlv{Manual}

Manual is a controller that allow to move a mirror in the four
directions. It require a \textbf{Mirror} and a \textbf{Centroid}. This
last one is from a \textbf{Frame} inside a \textbf{Camera} or from a
\textbf{PSD}. For semplicity here I provide an example using a
\textbf{IDS} camera and a \textbf{STT254} mirror.

First you need to include the header file and use the namespace.

\begin{lstlisting}[language=c++, gobble=2]
  #include <PAT/Controller/Manual.h>
  #include <PAT/Mirror/STT254.h>
  #include <PAT/Camera/IDS.h>
  using namespace PAT::Controller;
  using namespace PAT::Camera;
  using namespace PAT::Mirror;
\end{lstlisting}


When you istantiate the \textbf{Manual} you must pass a \textbf{Mirror}
and a \textbf{Centroid} istances.

\begin{lstlisting}[language=c++, gobble=2]
  std::shared_ptr<Camera> camera      = std::make_shared<IDS>();
  std::shared_ptr<PAT::KPA101> kpa101 = std::make_shared<PAT::KPA101>();
  std::shared_ptr<Mirror> mirror      = std::make_shared<STT254>(kpa101);
  std::shared_ptr<Manual> manual      = std::make_shared<Manual>(mirror, camera->getFrame()->getCentroid());
\end{lstlisting}


To move in one of the four directions you can use this code.

\begin{lstlisting}[language=c++, gobble=2]
  manual->move(Direction::up);
  manual->move(Direction::down);
  manual->move(Direction::left);
  manual->move(Direction::right);
\end{lstlisting}


To choose how much move you need to set times (milliseconds) and
voltages (volts) in the four directions. You can also control the
current values.

\begin{lstlisting}[language=c++, gobble=2]
  if(manual->getManualVoltage(Direction::up) > 100)
    manual->setManualVoltage(1, Direction::up);
  if(manual->getManualTime(Direction::up) > 100)
    manual->setManualTime(1, Direction::down);
\end{lstlisting}


You can also invert or check the movement directions in the two axes.
You can also swap the two axes.

\begin{lstlisting}[language=c++, gobble=2]
  if(!manual->getFlippedX())
    manual->setFlippedX(true);
  else
    manual->setFlippedX(false);
  
  if(!manual->getFlippedY())
    manual->setFlippedY(true);
  else
    manual->setFlippedY(false);
  
  if(!manual->getSwappedXY())
    manual->setSwappedXY(true);
  else
    manual->setSwappedXY(false);
\end{lstlisting}


\lvlv{ManualWidget}

The \textbf{ManualWidget} class is a widget for the Manual controller.
Here I present the an example consistent with the Manual one. The
include code is the following.

\begin{lstlisting}[language=c++, gobble=2]
  #include <QApplication>
  #include <PAT/Controller/gui/ManualWidget.h>
  #include <PAT/Mirror/STT254.h>
  #include <memory>
  #include <PAT/Camera/IDS.h>
  using namespace PAT::Controller;
  using namespace PAT::Camera;
  using namespace PAT::Mirror;
  
\end{lstlisting}


This is a simple \textbf{main} example.

\begin{lstlisting}[language=c++, gobble=2]
  int main(int argc, char *argv[]) {
    QApplication a(argc, argv);
    ManualWidget manualWidget;
    std::shared_ptr<Camera> camera      = std::make_shared<IDS>();
    std::shared_ptr<PAT::KPA101> kpa101 = std::make_shared<PAT::KPA101>();
    std::shared_ptr<Mirror> mirror      = std::make_shared<STT254>(kpa101);
    std::shared_ptr<Manual> manual      = std::make_shared<Manual>(mirror, camera->getFrame()->getCentroid());
    manualWidget.setController(manual);
    manualWidget.show();
    camera->start();
    return a.exec();
  }
\end{lstlisting}


If you want instead integrate the widget into another GUI (layout) you
can use this code.

\begin{lstlisting}[language=c++, gobble=2]
  QWidget *window                     = new QWidget();
  QVBoxLayout *mainLayout             = new QVBoxLayout();
  ManualWidget *manualWidget          = new ManualWidget();
  
  std::shared_ptr<Camera> camera      = std::make_shared<IDS>();
  std::shared_ptr<PAT::KPA101> kpa101 = std::make_shared<PAT::KPA101>();
  std::shared_ptr<Mirror> mirror      = std::make_shared<STT254>(kpa101);
  std::shared_ptr<Manual> manual      = std::make_shared<Manual>(mirror, camera->getFrame()->getCentroid());
  
  manualWidget->setPSD(manual);
  mainLayout->addWidget((QWidget *)manualWidget);
  window->setLayout(mainLayout);
  window->show();
\end{lstlisting}


There are also these three signal functions that they are used to
propagate the click of their respective buttons in other controllers.

\begin{lstlisting}[language=c++, gobble=2]
  void chkFlipXchanged(int arg1);
  void chkFlipYchanged(int arg1);
  void chkSwapXYchanged(int arg1);
\end{lstlisting}


\lvlv{CalibrationConfig}

This structure contains the main parameters for the calibration. The
parameters are: the four voltages, one per direction; the time durations
of the movements described by min-max-increment; the inversion or swap
of the axes. The include code is the following.

\begin{lstlisting}[language=c++, gobble=2]
  #include <PAT/Controller/CalibrationConfig.h>
\end{lstlisting}


You can also store/unstore from a file these parameters.

\begin{lstlisting}[language=c++, gobble=2]
  CalibrationConfig parameters;
  QString filename = "test.txt";
  parameters.write(filename);
  parameters.read(filename);
\end{lstlisting}


\lvlv{CalibrationAnalysis}

This is a support structure to take the data from a file and fit them
using a linear fit. The include code is the following.

\begin{lstlisting}[language=c++, gobble=2]
  #include <PAT/Controller/CalibrationAnalysis.h>
\end{lstlisting}

This structure can analyze file data generated by \textbf{Calibrated}
controller.

\begin{lstlisting}[language=c++, gobble=2]
  CalibrationAnalysis analysis;
  QString filename = "test.txt";
  analysis.read(filename);
\end{lstlisting}

\lvlv{Calibrated}

Calibrated is a controller that allow to move a mirror in the four
directions. It is able to calibrate itself using linear fit. It also
contain a One shot controller that in one frame it try to align the
centroid to the target point. It require a \textbf{Mirror}, a
\textbf{Centroid} and a target point. The \textbf{Centroid} and the
calibrated point are from a \textbf{Frame} inside a \textbf{Camera} or
from a \textbf{PSD}. For semplicity here I provide an example using a
\textbf{IDS} camera and a \textbf{STT254} mirror.

First you need to include the header file and use the namespace.

\begin{lstlisting}[language=c++, gobble=2]
  #include <PAT/Controller/Calibrated.h>
  #include <PAT/Mirror/STT254.h>
  #include <PAT/Camera/IDS.h>
  using namespace PAT::Controller;
  using namespace PAT::Camera;
  using namespace PAT::Mirror;
\end{lstlisting}

When you istantiate the \textbf{Calibrated} you must pass a
\textbf{Mirror} and a \textbf{Centroid} istances.

\begin{lstlisting}[language=c++, gobble=2]
  std::shared_ptr<Camera> camera         = std::make_shared<IDS>();
  std::shared_ptr<PAT::KPA101> kpa101    = std::make_shared<PAT::KPA101>();
  std::shared_ptr<Mirror> mirror         = std::make_shared<STT254>(kpa101);
  std::shared_ptr<Calibrated> calibrated = std::make_shared<Calibrated>(mirror, camera->getFrame()->getCentroid(), camera->getFrame()->getTarget());
\end{lstlisting}

To calibrate the controller you need first to collect the data. To do so
you need to choose a configuration and a file name. You can use this
code as example of command you require.

\begin{lstlisting}[language=c++, gobble=2]
  auto upVoltage     = calibrated->getInputVoltage(Direction::up);
  auto downVoltage   = calibrated->getInputVoltage(Direction::down);
  auto leftVoltage   = calibrated->getInputVoltage(Direction::left);
  auto rightVoltage  = calibrated->getInputVoltage(Direction::right);
  upVoltage.value    = 2;
  downVoltage.value  = 2;
  leftVoltage.value  = 2;
  rightVoltage.value = 2;
  calibrated->setInputVoltage(upVoltage, Direction::up);
  calibrated->setInputVoltage(downVoltage, Direction::down);
  calibrated->setInputVoltage(leftVoltage, Direction::left);
  calibrated->setInputVoltage(rightVoltage, Direction::right);
  
  auto timeMin  = calibrated->getTimeIntervalMin();
  auto timeInc  = calibrated->getTimeIntervalInc();
  auto timeMax  = calibrated->getTimeIntervalMax();
  timeMin.value = 1;
  timeInc.value = 0.5;
  timeMax.value = 5;
  calibrated->setTimeIntervalMin(timeMin);
  calibrated->setTimeIntervalInc(timeInc);
  calibrated->setTimeIntervalMax(timeMax);
  
  auto nPoints = calibrated->getNPoints();
  nPoints.value = 10;
  calibrated->setNPoints(nPoints);
  
  calibrated->setFlippedX(true);
  calibrated->setFlippedY(false);
  calibrated->setSwappedXY(false);
  
  calibrated->autoFileName();
\end{lstlisting}

After selected the config you can start/stop to collect the data.

\begin{lstlisting}[language=c++, gobble=2]
  calibrated->startTakeCalibrationPoints();
  calibrated->stopTakeCalibrationPoints();
\end{lstlisting}

After collected the data you can analyze it and store the result in
cache.

\begin{lstlisting}[language=c++, gobble=2]
  calibrated->analyzeCalibrationPoints();
\end{lstlisting}

To conclude the process, when you are happy with the cached data you can
use it as configuration to the calibrated movement. Remember to select
the time threshold in milliseconds. If the calibrated movement time is
higher than threshold the controller use an high voltage else a low
voltage.

\begin{lstlisting}[language=c++, gobble=2]
  calibrated->useCurrentCalibration(VoltageType::low);
  calibrated->useCurrentCalibration(VoltageType::high);
  auto threshold = calibrated->getThreshold();
  threshold.value = 2;
  calibrated->setThreshold(threshold);
\end{lstlisting}

Completed this process you can use the directional movement or the One
Shot controller.

\begin{lstlisting}[language=c++, gobble=2]
  auto deltaPx = calibrated->getDeltaPixel();
  deltaPx.value = 50;
  calibrated->setDeltaPixel(deltaPx);
  calibrated->moveAuto(Direction::up);
  
  calibrated->startOneShotController();
  calibrated->stopOneShotController();
\end{lstlisting}

\lvlv{CalibratedWidget}

The \textbf{CalibratedWidget} class is a widget for the Calibrated
controller. Here I present the an example consistent with the Calibrated
one. The include code is the following.

\begin{lstlisting}[language=c++, gobble=2]
  #include <QApplication>
  #include <PAT/Controller/gui/CalibratedWidget.h>
  #include <PAT/Mirror/STT254.h>
  #include <memory>
  #include <PAT/Camera/IDS.h>
  using namespace PAT::Controller;
  using namespace PAT::Camera;
  using namespace PAT::Mirror;
\end{lstlisting}

This is a simple \textbf{main} example.

\begin{lstlisting}[language=c++, gobble=2]
  int main(int argc, char *argv[]) {
    QApplication a(argc, argv);
    CalibratedWidget calibratedWidget;
    std::shared_ptr<Camera> camera      = std::make_shared<IDS>();
    std::shared_ptr<PAT::KPA101> kpa101 = std::make_shared<PAT::KPA101>();
    std::shared_ptr<Mirror> mirror      = std::make_shared<STT254>(kpa101);
    std::shared_ptr<Calibrated> calibrated  = std::make_shared<Calibrated>(mirror, camera->getFrame()->getCentroid(), camera->getFrame()->getTarget());
    calibratedWidget.setController(calibrated);
    calibratedWidget.show();
    camera->start();
    return a.exec();
  }
\end{lstlisting}

If you want instead integrate the widget into another GUI (layout) you
can use this code.

\begin{lstlisting}[language=c++, gobble=2]
  QWidget *window                     = new QWidget();
  QVBoxLayout *mainLayout             = new QVBoxLayout();
  CalibratedWidget *calibratedWidget  = new CalibratedWidget();
  
  std::shared_ptr<Camera> camera         = std::make_shared<IDS>();
  std::shared_ptr<PAT::KPA101> kpa101    = std::make_shared<PAT::KPA101>();
  std::shared_ptr<Mirror> mirror         = std::make_shared<STT254>(kpa101);
  std::shared_ptr<Calibrated> calibrated = std::make_shared<Calibrated>(mirror, camera->getFrame()->getCentroid());
  
  calibratedWidget->setPSD(calibrated);
  mainLayout->addWidget((QWidget *)calibratedWidget);
  window->setLayout(mainLayout);
  window->show();
\end{lstlisting}

There are also these three signal functions that they are used to
propagate the click of their respective buttons in other controllers.

\begin{lstlisting}[language=c++, gobble=2]
  void chkFlipXchanged(int arg1);
  void chkFlipYchanged(int arg1);
  void chkSwapXYchanged(int arg1);
\end{lstlisting}

\lvlv{DistancePlotterWidget}

The \textbf{DistancePlotterWidget} class is a widget for plot the
distance between a centroid and a point. This widget is used in
combination with \textbf{CalibratedWiget} and \textbf{ManualWidget}. The
include code is the following.

\begin{lstlisting}[language=c++, gobble=2]
  #include <QApplication>
  #include <PAT/Controller.h>
  #include <PAT/Centroid.h>
  #include <PAT/Utils/Point.h>
  #include <memory>
  #include <thread>
  #include <stdlib.h>
  #include <QDebug>
  using namespace PAT::Controller;
\end{lstlisting}

This is a simple \textbf{main} example.

\begin{lstlisting}[language=c++, gobble=2]
  int main(int argc, char *argv[]) {
    QApplication a(argc, argv);
    DistancePlotterWidget distancePlotterWidget;
    std::shared_ptr<PAT::Centroid> centroid(new PAT::Centroid());
    std::shared_ptr<PAT::Utils::BoundedParameter<PAT::Utils::Circle<int>>> target(new PAT::Utils::BoundedParameter<PAT::Utils::Circle<int>>({0, 1000, 1, 500}, {0, 1000, 1, 500}));
    double frameRate = 30;
    distancePlotterWidget.setCentroidAndTarget(centroid, target);
    std::thread t1([&]() {
      for (size_t i = 0; i < 50000; i++) {
        centroid->x = (double)rand() / RAND_MAX * 1000;
        centroid->y = (double)rand() / RAND_MAX * 1000;
        emit centroid->newValue();
        std::this_thread::sleep_for(std::chrono::milliseconds((int)round(1000 / frameRate)));
      }
    });
    distancePlotterWidget.show();
    t1.detach();
    return a.exec();
  }
\end{lstlisting}
