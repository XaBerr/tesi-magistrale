\lvliii{Centroid}

This project contains the centroid class. This class contains the
centroid information shared between the sensor and the controller in the
PAT-system.

\textbf{Requirements}

\includegraphics[scale=0.7]{img/shilds/cpp.png}
\includegraphics[scale=0.7]{img/shilds/cmake.png}
\includegraphics[scale=0.7]{img/shilds/git.png}
\includegraphics[scale=0.7]{img/shilds/doxygen.png}
\includegraphics[scale=0.7]{img/shilds/sphinx.png}
\includegraphics[scale=0.7]{img/shilds/win.png}
\includegraphics[scale=0.7]{img/shilds/mac.png}
\includegraphics[scale=0.7]{img/shilds/linux.png}

\lvliv{Generality}

\lvlv{Import}

Import as an external library into your project by copy-paste the
following lines in your \texttt{config.json}.

\begin{lstlisting}[language=javascript, gobble=2]
  {
    "name"     : "PATCentroid",
    "path"     : "gitlab.dei.unipd.it/PAT/Centroid.git",
    "tag"      : "HEAD",
    "available": "YES",
    "getGui"   : "NO"
  }
\end{lstlisting}

\lvlv{Prerequisites}

The following libraries are auto fetched from the gitlab.dei.unipd.it
host (ask the owner of this repo to become a member):

\begin{itemize}
  \tightlist
  \item
        \href{https://gitlab.dei.unipd.it/PAT/Utils.git}{Utils} 1.0.0
\end{itemize}

These other libraries need to be installed manually in your system:

\begin{itemize}
  \tightlist
  \item
        \href{https://www.qt.io/}{Qt} 5.14.2
\end{itemize}

The library documentation is generated through
\href{http://www.doxygen.nl/download.html}{Doxygen 1.8.13}. Additional
documentation in the \texttt{index} folder is generated through the
\href{https://www.anaconda.com/products/individual}{python3} package
\href{https://www.sphinx-doc.org/en/master/}{Sphinx} using the following
extensions (which you can install through pip3):

\begin{itemize}
  \tightlist
  \item
        \href{https://pypi.org/project/Sphinx/}{Sphinx 3.0.2}
  \item
        \href{https://sphinx-rtd-theme.readthedocs.io/en/stable/}{Sphinx read
          the doc theme} (to use the read the doc theme for html documentation)
  \item
        \href{https://pypi.org/project/breathe/}{Breathe} (to use the xml
        output of doxygen)
  \item
        \href{https://pypi.org/project/sphinx-markdown-builder/}{Sphinx-markdown-builder}
        (to generate the markdown version for gitlab wiki)
\end{itemize}

\texttt{pip3\ install\ Sphinx\ sphinx\_rtd\_theme\ breathe\ sphinx-markdown-builder}

\lvliv{Usage}

Centroid give the possibility to store the point information with
\textbf{x} and \textbf{y} using a
\textbf{Point\textless double\textgreater{}} structure, and also the
\textbf{instensity} information.

It also contains a signal \textbf{newValue} that it could be emitted
when a new value is ready.

First you need to include the header file and use the namespace.

\begin{lstlisting}[language=c++, gobble=2]
  #include <PAT/Centroid.h>
  using namespace PAT;
\end{lstlisting}

Then you can define the variable ready to be used.

\begin{lstlisting}[language=c++, gobble=2]
  Centroid centroid(10, 20, 100);
  centroid.x = 10;
  centroid.y = 20;
  centroid.instensity = 100;
\end{lstlisting}

If you need to emit the signal use this function.

\begin{lstlisting}[language=c++, gobble=2]
  emit centroid.newValue();
\end{lstlisting}

\lvliv{UML}
\img{centroid}{Centroid UML.}
